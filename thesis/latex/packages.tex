\usepackage[T1]{fontenc}
\usepackage[utf8]{inputenc}
\usepackage{lmodern}

\usepackage{makeidx}
    % Needed to build the index.

\usepackage{amsmath}
    % Brings the align environment for lining up equations on the =.

\usepackage{amssymb}
    % For the number set symbols.

\usepackage{graphicx}
    % Can't have pictures/photos/figures from files without that.
    % Note that vanilla latex will only accept eps.
    
\usepackage{subfigure}
    % Lets me have labels such as a) b) c) inside a figure environment.

\usepackage[font=footnotesize, labelfont=bf]{caption}
    % So that I can add long captions under figures.
    % It provides me with \caption* that does not appear in the list of
    % figures but formats the text like \caption does.  It also lets me
    % use line breaks in the caption, and even bullet/enumerated lists.

\usepackage{color}
    % For rendering text is eps_tex files produced by InkScape, even
    % if the text is black.

\usepackage{booktabs}
    % For professional-looking tables.
    % Brings the \toprule, \midrule and \bottomrule.
    % Remember not to use vertical rules in tables: they look cheap.

\usepackage[version=3]{mhchem}
    % For chemical formulas.
    % Brings \ce.ormulas.

\usepackage[mediumspace,mediumqspace,squaren]{SIunits}
    % Otherwise I can't write the mu symbol for micrometers.
    % It also brings me the \degree symbol, woo!
    % No decibel though, I need to make this one myself.

\usepackage{stmaryrd}
    % For the llbracket and rrbracket to denote integer intervals.

\usepackage{algorithmic}
    % For pseudocode.
\usepackage[chapter]{algorithm}
    % To float algorithms.

\usepackage{varioref}
    % For fancier references that also tell the page number.
    
\usepackage{todonotes}
    % Big post-its, handy while writing.
    
\usepackage[]{biblatex}
    % Fantastic bibliography manager that I'll use just for its
    % \citetitle command.

\newcommand{\decibel}{dB}
\newcommand{\equaldef}{\stackrel{\text{\tiny def}}{=}}
\newcommand{\transp}{^T}
\newcommand{\norm}[1]{\left\| #1 \right\|}
\newcommand{\abs}[1]{\left| #1 \right|}
