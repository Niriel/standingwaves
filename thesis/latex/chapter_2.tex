In this chapter, I present a solver and prove that it works, at least qualitatively, by using typical networks.

ALL THEORY

\section{Solver}

Maths go here.

\section{Generic networks}

Scattering matrix of typical networks, with parameters.

Some of these parameters may not be super physical, but are here to kinda absorb the imperfections and the mismatch when fitting.

For the grid I use Houde \cite{houde_2001} because there is nothing simpler, but for the other networks I keep it as simple as I can.  Grids being VERY complex to model, I might spend a lot of time and space on that one.

So, in the end, it's a list of scattering matrices.  It would be nice to augment each system with plots showing the effect of each parameter.

\section{Simple systems}

The idea is to show that it works and makes sense.

I must illustrate the most simple cavity.  I show it creates a standing wave pattern, I show its FFT, prove it's consistent with the length of the cavity and its refractive index.

Then I introduce a thin film beam splitter (that's already a heterodyne telescope when we think of it).  I show that it adds a slope to the previous pattern.  I show that it's actually a very slow standing wave, the period of which is consistent with infinite reflections inside the thin film.  The FFT should show that there are two picks close to the cavity-period: they are due to the fact that there are now two cavities: one using reflection on the near side, one on the far side of the beam splitter.

All of this makes sense, we are confident that the model does its job properly.
