In this chapter, I present a solver and prove that it works, at least qualitatively, by using typical networks.

ALL THEORY

\section{Solver}

\subsection{Simplifications}

Plane wave.
Single mode.

Not only a list, I must justify them.
Anything below -40 dB can pretty much be ignored \todo{why?}.
What does not couple is lost, so just model the higher modes with a loss parameter.
This is going to work unless higher modes can come back to the fundamental mode, but that should be a very small effect.



\subsection{Jones calculus in 3D}

\subsection{Scattering matrices of Jones matrices}

%=============================================================================
\subsection{Solving a system}

%-----------------------------------------------------------------------------
\paragraph{Theory}

We call $S$ the scattering matrix of the entire system.
\begin{equation*}
    S =
    \begin{pmatrix}
        S_{1, 1} & S_{1, 2} & \cdots & S_{1, n} \\
        S_{2, 1} & S_{2, 2} & \cdots & S_{2, n} \\
        \vdots   & \vdots   & \ddots & \vdots \\
        S_{n, 1} & S_{n, 2} & \cdots & S_{n, n}
    \end{pmatrix}
\end{equation*}
\begin{equation*}
    b = Sa
\end{equation*}
S contains information about all the ports in the system: the ports that are open to the outside world, but also the ports hidden within the system because the networks are coupled to each other.
Here, $n$ is the number of ports of $S$, equal to the sum of the number of ports of all the networks constituting the system.

Some of the networks constituting the system are coupled by one port.
If two networks G and H are coupled by their port $g$ and $h$, then the output of $g$ is the input of $h$ and the output of $h$ is the input of $g$, as illustrated by Equation~\eqref{eq:coupled_inputs_outputs}.
\begin{equation}
    \left\lbrace
    \begin{aligned}
        b_h &= a_g \\
        b_g &= a_h
    \end{aligned}
    \right.
    \label{eq:coupled_inputs_outputs}
\end{equation}

Ports that are coupled are called ``inside port'', those that are not coupled are called ``outside port''.
If we note~$a^o$ the inputs from the outside world,~$b^o$ the outputs to the outside world,~$a^i$ the inputs inside the system and~$b^i$ the outputs outside the system, then we can reorder the rows and columns of the vectors~$a$,~$b$ and the matrix~$S$ to put together the inside ports and the outside ports: 
\begin{equation}
    \binom{b^o}{b^i} =
    \begin{pmatrix}
        S^{oo} & S^{oi} \\
        S^{io} & S^{ii} \\
    \end{pmatrix}
    \binom{a^o}{a^i}
    \label{eq:S_outside_inside}
\end{equation}
Only~$a^o$ is known: this corresponds to the input to the system and we control it.
$a^i$,~$b^i$ and~$b^o$ are unknown.
$b^o$ is the result we are looking for.
$b^i$ and~$a^i$ are nice to have as they allow us to see what is happening inside the system.

How are~$a^i$ and~$b^i$ related?
Equation~\eqref{eq:coupled_inputs_outputs} indicates that each element of $a^i$ is equal to an element of $b^i$, therefore $a^i$ can be obtain by reordering $b^i$ and vice-versa.
\index{permutation matrix}In other words, there exists a permutation matrix~$P$ such that
\begin{equation}
    a^i = P b^i \text{.} \label{eq:relation_ai_bi}
\end{equation}

To get $b_o$, the first step is to compute~$b_i$ from~$a_o$.
\begin{align}
    b^i &= S^{io}a^o + S^{ii}a^i \notag \\
    b^i &= S^{io}a^o + S^{ii}Pb^i \notag \\
    (I - S^{ii}P)b^i &= S^{io}a^o \notag \\
    b^i &= (I - S^{ii}P)^{-1} S^{io}a^o \label{eq:compute_bi}
\end{align}
where~$I$ is the identity matrix with a dimension equal to that of~$S^{ii}P$.

The second step is to compute~$b^o$ from~$a^o$ and~$b^i$.
\begin{align}
    b^o &= S^{oo}a^o + S^{oi}a^i \notag \\
    b^o &= S^{oo}a^o + S^{oi}Pb^i \label{eq:compute_bo}
\end{align}

These two steps solve the system: from the inputs~$a^o$ we can compute the outputs~$b^o$ of the whole system.
This method also tells gives us $a^i$ and $b^i$ which are two equivalent ways (Equation~\eqref{eq:relation_ai_bi}) of looking at what is happening inside the system.

%-----------------------------------------------------------------------------
\paragraph{Implementation}
Although the mathematics are quite simple, the implementation requires careful book keeping of the many indices of the many matrices.

\begin{itemize}
    \item 
The system contains $N$ networks.
    \item 
Each network is modeled with a~$n_i$--by--$n_i$ scattering matrix, where~$n_i$ is the number of ports the network $i$, with $i \in \llbracket 1, N \rrbracket$.
    \item 
The whole system is modeled with a~$n$--by--$n$ scattering matrix, where~$n$ is the number of ports of the whole system, with $n = \sum_{i=1}^N n_i$.
\end{itemize}

Each network $i$ comes with ports that are ordered from 1 to $n_i$.
The networks themselves are ordered since they are indexed from 1 to $N$.
Therefore, there is a natural way of numbering the ports of the whole system.
The port~$j$ of the network~$i$ corresponds to the port~$k$ of the whole system, where
\begin{equation}
    k = \sum_{m=0}^{j - i}n_m + j \label{eq:port_numbering}
\end{equation}

With that information, it is trivial to fill the scattering matrix of the whole system, $S$, with values from the scattering matrices from each network $S_i$.
$S$ contains mostly zeros, except for blocks on its diagonal containing copies of each~$S_i$.
For example, Equation~\eqref{eq:block_scattering} shows what $S$ looks like for a system containing two networks $S_1$ and $S_2$.
\begin{equation}
    S =
    \begin{pmatrix}
        {S_1}_{1, 1} & \cdots & {S_1}_{1, n_1} &
        0 & \cdots & 0
        \\
        \vdots & \ddots &\vdots &
        \vdots & \ddots &\vdots 
        \\
        {S_1}_{n_1, 1} & \cdots & {S_1}_{n_1, n_1} &
        0 & \cdots & 0
        \\
        0 & \cdots & 0 &
        {S_2}_{1, 1} & \cdots & {S_2}_{1, n_2}
        \\
        \vdots & \ddots &\vdots &
        \vdots & \ddots &\vdots
        \\
        0 & \cdots & 0 &
        {S_2}_{n_2, 1} & \cdots & {S_2}_{n_2, n_2}
    \end{pmatrix}
    \label{eq:block_scattering}
\end{equation}
Note that the number of zeros increases with~$N^2$ while the number of non-zeros elements around the diagonal increases with~$N$.
This remark has no importance from a purely mathematical point of view but is very important for the implementation of the solving algorithm.
\index{space matrix}Indeed, matrices that contain many zeros are called ``sparse matrices'' and many algorithms exist that have been optimized for dealing with them.
If we use such an optimized algorithm, then the matrix inversion that appears in Equation~\eqref{eq:compute_bi} will execute in $O(N)$ instead of $O(N^2)$ (using the ``big O notation'').

To continue, we need to know which ports are coupled to which ports.
We can represent the coupling with a matrix~$C$.
$C$ contains 0 for the ports that are not coupled, 1 for those that are.
A sane description of the coupling has the following properties:
\begin{itemize}
    \item A port is not coupled to itself; the diagonal of~$C$ contains only zeros.
    \item A port is not coupled to more than one port; there is at most one 1 on a row or column of~$C$.
    \item If the port~$i$ is coupled to the port~$j$, then the port~$j$ is coupled to the port~$i$;~$C$ is symmetric.
    \item For every port in the system, there exist a path that connects it to an outside port; this cannot be ensured by looking only at~$C$ and requires using graph theory.
\end{itemize}
It is possible to use graph theory to detect some errors in the coupling matrix, but it is not required for the solver to work.
From now on, we assume that the data is healthy.

Unless we are lucky, the system-level scattering matrix (that we build from the network-level scattering matrices) is not organized into inside/outside sectors like shown on Equation~\eqref{eq:S_outside_inside}.
However, there exists a permutation matrix~$P$ such as
\begin{equation}
    S' = P S P^{-1}
\end{equation}
with $S'$ having the form of Equation~\eqref{eq:S_outside_inside}.
That permutation matrix $P$ is related to the coupling matrix $C$.


\section{Generic networks}

Scattering matrix of typical networks, with parameters.

Some of these parameters may not be super physical, but are here to kinda absorb the imperfections and the mismatch when fitting.

For the grid I use Houde \cite{houde_2001} because there is nothing simpler, but for the other networks I keep it as simple as I can.  Grids being VERY complex to model, I might spend a lot of time and space on that one.

So, in the end, it's a list of scattering matrices.  It would be nice to augment each system with plots showing the effect of each parameter.

\section{Simple systems}

The idea is to show that it works and makes sense.

I must illustrate the most simple cavity.  I show it creates a standing wave pattern, I show its FFT, prove it's consistent with the length of the cavity and its refractive index.

Then I introduce a thin film beam splitter (that's already a heterodyne telescope when we think of it).  I show that it adds a slope to the previous pattern.  I show that it's actually a very slow standing wave, the period of which is consistent with infinite reflections inside the thin film.  The FFT should show that there are two picks close to the cavity-period: they are due to the fact that there are now two cavities: one using reflection on the near side, one on the far side of the beam splitter.

All of this makes sense, we are confident that the model does its job properly.
