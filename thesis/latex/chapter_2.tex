\section{Principle}
Our method is inspired from laser and circuit theory: we deal with discrete inputs and outputs rather than, for example, a spatial description of the fields.
A system like the Focal Plane Unit of HIFI comprises a few dozens of optical elements~\cite{jackson2002hifi}, referred in the literature as ``networks''~\cite{siegman1986lasers}: wire grid polarizers, roof-top mirrors, horns, attenuators, free space, all having one or more input and output.
All these networks interact together two by two since there exists at least one optical path between each pair or networks.
The number of interactions in the system increases faster than the square of the number of networks.
One way to solve a system's reaction to a set of input would be to propagate the inputs from network to network, back and forth, until the fields appear to converge to a steady state.
Instead, we propose a method to that directly provides the steady state.

%#############################################################################
\section{Solver}

%=============================================================================
\subsection{Simplifications}

Plane wave.
Single mode.

Not only a list, I must justify them.
Anything below -40 dB can pretty much be ignored \todo{why?}.
What does not couple is lost, so just model the higher modes with a loss parameter.
This is going to work unless higher modes can come back to the fundamental mode, but that should be a very small effect.





%=============================================================================
\subsection{Phasors}
\index{phasor}
\Cref{eq:e_z_t_real} is an expression of the amplitude of the electric field of a plane wave propagating along an axis~$\vect{z}$.
\begin{equation}
   e(z, t) = E \cos(kz - \omega t + \varphi)
   \label{eq:e_z_t_real}
\end{equation}
$t$ is the time coordinate and~$z$ the space coordinate.
$\omega$,~$k$ are constant related to the frequency and the propagation medium.
$\varphi$ is related to initial conditions and the choice of reference frame.
$E$ is the amplitude of the electric field given for example in volts per meter or newtons per coulomb.
It is illustrated \cref{fig:plane_wave_propagation} as a function of~$z$ for a few values of~$t$.
\begin{figure}[hbp]
    \centering
    \includegraphics{plane_wave_propagation}
    \caption {\label{fig:plane_wave_propagation}Electric field amplitude of a wave of frequency \SI{500}{\giga\hertz} and amplitude \SI{1}{\volt\per\meter} propagating in vacuum (refractive index 1).}
\end{figure}

\subsubsection{Polar complex notation.}
It is easier to go on by using polar complex notations.
Euler's formula, given in \cref{eq:euler_formula}, relates the trigonometric function $\cos$ to a complex exponential.
\begin{equation}    
    \exp(j\theta) = \cos(\theta) + j\sin(\theta) \label{eq:euler_formula}
\end{equation}
In~\cref{eq:euler_formula}, $j$ represents the imaginary unit such that $j^2=-1$.
If we rewrite the expression of~$e$ as complex in polar coordinates, we have \cref{eq:e_z_t_complex}.
\begin{align}
   \hat{e}(z, t) &= E \exp(j(kz - \omega t + \varphi))
   \label{eq:e_z_t_complex}
   \\
   e(z, t) &= \Re(\hat{e}(z, t))
   \label{eq:e_z_t_real_complex}
\end{align}
The hat above the $e$ signals that $\hat{e}$ is the complex equivalent to $e$.
As shown by \cref{eq:e_z_t_real_complex}, $e$ and $\hat{e}$ do not have the same value (one is real, the other is complex);
however, both describe the same physical phenomenon.

By applying the exponentiation identity \cref{eq:exponentiation_identity_plus},
\cref{eq:e_z_t_complex} can also be written as \cref{eq:e_z_t_complex_factorized}.
\begin{equation}
    \exp(a+b) = \exp(a) \exp(b)
    \label{eq:exponentiation_identity_plus}
\end{equation}
\begin{equation}
    \hat{e}(z, t)
    =
    E \exp(jkz) \exp(-j\omega t) \exp(j\varphi)
    \label{eq:e_z_t_complex_factorized}
\end{equation}

In the rest of this book, we will only ever use the polar complex notation, unless stated otherwise.
To simplify the notations, we will not place a hat symbol above the complex equivalent to real quantities.

\subsubsection{No need to carry the time.}
Linear networks do not modify the frequency~$f$ of the waves that propagate through them.
Since~$\omega = 2 \pi f$, $\omega$ is constant in the entire system.
At any instant $t$, the value of $t$ is a constant.
\Crefrange{eq:factor_out_omegat_0}{eq:factor_out_omegat_4} show that if $f$ is a linear function, then the time-dependant factor $\exp(j\omega t)$ is actually a constant that we can safely ignore.
\begin{align}
    e'(z, t)
    &=
    f(e(z, t))
    \label{eq:factor_out_omegat_0}
    \\
    E' \exp(j(k'z - \omega t + \varphi'))
    &=
    f\Big(
        E \exp(j(kz - \omega t + \varphi))
    \Big)
    \label{eq:factor_out_omegat_1}
    \\
    E' \exp(j(k'z + \varphi'))\exp(-j \omega t)
    &=
    f\Big(
        E \exp(j(kz + \varphi)) \exp(-j \omega t)
    \Big)
    \label{eq:factor_out_omegat_2}
    \\
    E' \exp(j(k'z + \varphi'))
    \,\cancel{
        \exp(-j \omega t)
    }
    &=
    f\Big(
        E \exp(j(kz + \varphi))
    \Big)
    \,\cancel{
        \exp(-j \omega t)
    }
    \label{eq:factor_out_omegat_3}
    \\
    E' \exp(j(k'z + \varphi'))
    &=
    f\Big(
        E \exp(j(kz + \varphi))
    \Big)
    \label{eq:factor_out_omegat_4}
\end{align}
It is therefore useless to carry the term~$\omega t$ in our equations.

\subsubsection{No need to carry the initial phase.}
The initial phase~$\varphi_0$ is given for the arbitrary point~$(z, t) = (0, 0)$.
Any phase $\varphi$ can be expressed as a deviation from the initial phase $\varphi_0$ as illustrated by \cref{eq:factor_out_phi0}.
\begin{equation}
    \varphi = \varphi_0 + \Delta \varphi
    \label{eq:factor_out_phi0}
\end{equation}
Like~$\omega$, $\varphi_0$ is constant throughout the whole system.
\Crefrange{eq:factor_out_phi0_0}{eq:factor_out_phi0_3} show that we can use that fact to safely ignore the absolute phase and work with relative phases only.
\begin{align}
    E' \exp(j(k'z + \varphi'))
    &=
    f\Big(
        E \exp(j(kz + \varphi))
    \Big)
    \label{eq:factor_out_phi0_0}
    \\
    E' \exp(j(k'z + \varphi_0+\Delta\varphi'))
    &=
    f\Big(
        E \exp(j(kz + \varphi_0+\Delta\varphi))
    \Big)
    \label{eq:factor_out_phi0_1}
    \\
    E' \exp(j(k'z +\Delta\varphi'))
    \,\cancel{
        \exp(j\varphi_0)
    }
    &=
    f\Big(
        E \exp(j(kz+\Delta\varphi))
    \Big)
    \,\cancel{\exp(j \varphi_0)}
    \label{eq:factor_out_phi0_2}
    \\
    E' \exp(j(k'z+\Delta\varphi'))
    &=
    f\Big(
        E \exp(j(kz+\Delta\varphi))
    \Big)
    \label{eq:factor_out_phi0_3}
\end{align}

\subsubsection{No need to carry the space.}
We are hardly ever interested in knowing the field at every location in space.
We concern ourselves mainly with a few and specific locations.
Typical locations include the surfaces of optical elements, such as mirrors, lenses, grids, horns, and the beginning and end of a propagation medium.

Each propagation medium has its own value for~$k$ and has a fixed pathlength~$\Delta z$.
Therefore, such a medium induces a constant phase shift of~$\Delta\phi = k\Delta z$ to the wave that propagates through it.

It is not important to know that the wave exits a propagation medium at~$z=z_0 + \Delta z$;
what matters is that the phase of the wave has changed by~$\Delta\phi$.
The effect matters more than the cause.
Like for $\varphi_0$, there is no use in keeping the absolute $kz_0$ around.
Instead we introduce a phase constant~$\Delta\phi$ for each position that interests us, as shown by \crefrange{eq:factor_out_kz0_0}{eq:factor_out_kz0_4}.
\begin{align}
    E' \exp(j(kz'+\Delta\varphi'))
    &=
    f\Big(
        E \exp(j(kz+\Delta\varphi))
    \Big)
    \label{eq:factor_out_kz0_0}
    \\
    E' \exp(j(k(z_0 + \Delta z')+\Delta\varphi'))
    &=
    f\Big(
        E \exp(j(k(z_0 + \Delta z)+\Delta\varphi))
    \Big)
    \label{eq:factor_out_kz0_1}
    \\
    E' \exp(j(k\Delta z'+\Delta\varphi')) \,\cancel{\exp(jkz_0)}
    &=
    f\Big(
        E \exp(j(k\Delta z+\Delta\varphi))
    \Big)
    \,\cancel{\exp(jkz_0)}
    \label{eq:factor_out_kz0_2}
    \\
    E' \exp(j(k\Delta z'+\Delta\varphi'))
    &=
    f\Big(
        E \exp(j(k\Delta z+\Delta\varphi))
    \Big)
    \label{eq:factor_out_kz0_3}
    \\
    E' \exp(j(\Delta\phi'+\Delta\varphi'))
    &=
    f\Big(
        E \exp(j(\Delta\phi+\Delta\varphi))
    \Big)
    \label{eq:factor_out_kz0_4}
\end{align}

\Cref{eq:factor_out_kz0_4} exhibits two terms of phase.
\begin{itemize}
    \item $\Delta\phi$ models the phase difference due to the propagation in a medium.
    \item $\Delta\varphi$ models the phase difference due to any other type of optical element.
\end{itemize}
This divide is arbitrary.
If we consider that a propagation medium is an optical element like mirrors or grids are, then we need only to keep $\Delta\varphi$ (\cref{eq:factor_out_phi}).
\begin{equation}
    E' \exp(j\Delta\varphi') = f\Big( E \exp(j\Delta\varphi) \Big)
    \label{eq:factor_out_phi}
\end{equation}

\subsubsection{No need to carry the exact amplitude.}
Our networks are all linear, which makes our system linear.
The output of a linear system being proportional to its input, we do not need to keep track of the exact input.
All that matters is that we can compute the coefficient of proportionality.
With \cref{eq:factor_out_E0}, we express any amplitude $E$ relatively to a reference amplitude $E_0 \ne 0$ with a proportionality factor $\epsilon$.
\begin{equation}
    E = \epsilon E_0 \label{eq:factor_out_E0}
\end{equation}
Using \cref{eq:factor_out_E0} in \cref{eq:factor_out_phi} allows us to simplify by $E_0$ (\crefrange{eq:factor_out_E0_0}{eq:factor_out_E0_1}).
\begin{align}
    \epsilon' \,\cancel{E_0} \exp(j\Delta\varphi')
    &=
    f\Big(
        \epsilon \,\cancel{E_0} \exp(j\Delta\varphi)
    \Big)
    \label{eq:factor_out_E0_0}
    \\
    \epsilon' \exp(j\Delta\varphi')
    &=
    f\Big(
        \epsilon \exp(j\Delta\varphi)
    \Big)
    \label{eq:factor_out_E0_1}
\end{align}



%-------------------------------------------------------------------------------
\subsubsection{Introducing phasors.}
With \cref{eq:factor_out_E0_1}, we are left with only two quantities:
\begin{itemize}
    \item a coefficient of proportionality between the amplitude at the output of a system and that at its input,
    \item a phase difference between the output and the input.
\end{itemize}

\todo[inline]{continue}

%=============================================================================
\subsection{Scattering matrices}
Scattering matrices \cite{siegman1986lasers} model the transfer of power or field from one side (port) of a network to another (\cref{fig:scattering_matrix_notations}).

\Cref{eq:scattering_matrix} presents the scattering matrix $S$ that models the relation between a vector of inputs $a$ and a vector of outputs $b$ for a $n$-ports network.
The diagonal contains the reflections terms and the rest contains the transmissions.
\begin{align}
    b &= S a
    &
    \begin{pmatrix}
        b_1\\
        b_2\\
        \vdots\\
        b_n
    \end{pmatrix}
    &=
    \begin{pmatrix}
        S_{1, 1} & S_{1, 2} & \cdots & S_{1, n} \\
        S_{2, 1} & S_{2, 2} & \cdots & S_{2, n} \\
        \vdots   & \vdots   & \ddots & \vdots   \\
        S_{n, 1} & S_{n, 2} & \cdots & S_{n, n}
    \end{pmatrix}
    \begin{pmatrix}
        a_1\\
        a_2\\
        \vdots\\
        a_n
    \end{pmatrix}
    \label{eq:scattering_matrix}
\end{align}

\begin{figure}[hbtp]
    \centering
    \input{\MyGraphicsPath scattering_matrix_notations.pdf_tex}
    \caption{A 4-ports network showing 4 inputs $a_i$ and four outputs $b_i$.  For example, this network could be a wire grid polarizer or a semi-transparent mirror, both acting as beam splitters.}%
    \label{fig:scattering_matrix_notations}
\end{figure}


%=============================================================================
\subsection{Jones calculus in 3D}

Jones matrices model the transfer of amplitude and phase from one polarization to another \cite{hecht2002optics}.
In~\cref{eq:jones_matrix_short}, the polarized input field $e_i$ is related to the polarized output field~$e_o$ by the Jones matrix $J$.
$e_i$ and $e_o$ are called ``Jones vectors''.
\begin{equation}
    % Small form.
    e_o = J e_i
    \label{eq:jones_matrix_short}
\end{equation}
The components of the Jones vectors are phasors: complex numbers that encode the amplitude and phase of the field along different axes.
The components of the Jones matrices are also complex numbers, these encode changes in amplitude and phase.

\Cref{eq:jones_matrix_2d} presents the traditional form of Jones vectors and matrices.
\begin{equation}
    % Jones matrix in 2D.
    \begin{pmatrix}
        e_{o, h}\\
        e_{o, v}
    \end{pmatrix}
    =
    \begin{pmatrix}
        J_{h, h}   &   J_{h, v} \\
        J_{v, h}   &   J_{v, v}
    \end{pmatrix}
    \begin{pmatrix}
        e_{i, h}\\
        e_{i, v}
    \end{pmatrix}
    \label{eq:jones_matrix_2d}
\end{equation}
The field is seen as the superposition of a horizontal and a vertical component, both linearly polarized.
The phase difference between these components determines the handedness and the ellipticity of the polarization of the field.
Although very useful for many applications, these Jones matrices and vectors have limitations.

First, the field is always expressed in a local reference frame: the horizontal and vertical directions are valid for one beam only and may change after any reflection or refraction.
Therefore, to completely describe a field, the Jones vector is not enough since one needs to keep track of the orientation of its reference frame with three angles, a quaternion or a rotation matrix.

Second, the horizontal and vertical directions are normal to each other and to the direction of propagation.
This limits us to modeling either transverse electric or transverse magnetic waves depending on whether the Jones vector represents the electric or the magnetic field;
this cannot model hybrid waves, in which both the electric and the magnetic field can have a component along the direction of propagation.
Therefore, this cannot model propagation in a birefringent material.
One way to accommodate this is to relax the constraint of orthogonality of the reference frame, allowing for the horizontal and vertical directions to have a component along the direction of propagation.
If we do this, then the full description of the field gets even more complicated as we need to carry along, not just the orientation of its reference frame, but the full description of that non-Cartesian reference frame.

Instead, we can express all the fields in a common global Cartesian reference frame, essentially extending Jones calculus from two to three dimensions, as illustrated in~\cref{eq:jones_matrix_3d}.
\begin{equation}
    % Jones matrix in 3D.
    \begin{pmatrix}
        e_{o, x}\\
        e_{o, y}\\
        e_{o, z}
    \end{pmatrix}
    =
    \begin{pmatrix}
        J_{x, x}   &   J_{x, y}   &   J_{x, z} \\
        J_{y, x}   &   J_{y, y}   &   J_{y, z} \\
        J_{z, x}   &   J_{z, y}   &   J_{z, z}
    \end{pmatrix}
    \begin{pmatrix}
        e_{i, x}\\
        e_{i, y}\\
        e_{i, z}
    \end{pmatrix}
    \label{eq:jones_matrix_3d}
\end{equation}
These Jones matrices can be combined with rotation matrices (built from Euler angles or quaternions) in order to represent networks in any orientation in space.

\subsubsection{Rotating Jones matrices.}
\label{sec:rotating_jones_matrices}
\begin{equation}
    J' = R J R^{-1}
    \label{eq:jones_rotation_using_inverse}
\end{equation}
\Cref{eq:jones_rotation_using_inverse} is correct, however it may not be the most practical to implement.
Indeed, inverting matrices is an expensive and unstable operation for a computer to perform%
\todo{citation needed}.
When we remember that the inverse of a rotation matrix is also its transpose%
\todo{citation needed},
then we get~\cref{eq:jones_rotation_using_transpose} which has the advantage of being computationally cheap and stable.
\begin{equation}
    J' = R J R\transp
    \label{eq:jones_rotation_using_transpose}
\end{equation}

Rotating the identity matrix has an interesting property shown by \cref{eq:jones_rotation_identity}.
\begin{equation}
    I' = R I R^{-1}
       = R R^{-1}
       = I
    \label{eq:jones_rotation_identity}
\end{equation}
If a Jones matrix is proportional to an identity matrix, then it is not modified by rotation.
We will use that property to save some computation, for example when propagating through homogeneous isotropic materials (\vref{sec:generic_networks_distance}).

%=============================================================================
\subsection{Combining scattering and Jones matrices}



%=============================================================================
\subsection{Solving a system}

%-----------------------------------------------------------------------------
\subsubsection{Theory}

We call $S$ the scattering matrix of the entire system.
\begin{equation*}
    S =
    \begin{pmatrix}
        S_{1, 1} & S_{1, 2} & \cdots & S_{1, n} \\
        S_{2, 1} & S_{2, 2} & \cdots & S_{2, n} \\
        \vdots   & \vdots   & \ddots & \vdots \\
        S_{n, 1} & S_{n, 2} & \cdots & S_{n, n}
    \end{pmatrix}
\end{equation*}
\begin{equation*}
    b = Sa
\end{equation*}
S contains information about all the ports in the system: the ports that are open to the outside world, but also the ports hidden within the system because the networks are coupled to each other.
Here, $n$ is the number of ports of $S$, equal to the sum of the number of ports of all the networks constituting the system.

Some of the networks constituting the system are coupled by one port.
If two networks G and H are coupled by their port $g$ and $h$, then the output of $g$ is the input of $h$ and the output of $h$ is the input of $g$, as illustrated by \cref{eq:coupled_inputs_outputs}.
\begin{equation}
    \left\lbrace
    \begin{aligned}
        b_h &= a_g \\
        b_g &= a_h
    \end{aligned}
    \right.
    \label{eq:coupled_inputs_outputs}
\end{equation}

Ports that are coupled are called ``inside port'', those that are not coupled are called ``outside port''.
If we note~$a^o$ the inputs from the outside world,~$b^o$ the outputs to the outside world,~$a^i$ the inputs inside the system and~$b^i$ the outputs outside the system, then we can reorder the rows and columns of the vectors~$a$,~$b$ and the matrix~$S$ to put together the inside ports and the outside ports: 
\begin{equation}
    \binom{b^o}{b^i} =
    \begin{pmatrix}
        S^{oo} & S^{oi} \\
        S^{io} & S^{ii} \\
    \end{pmatrix}
    \binom{a^o}{a^i}
    \label{eq:s_outside_inside}
\end{equation}
The four regions of~$S$ have the following physical meaning:
\begin{description}
    \item[$S^{oo}$:] from outside to outside: signal that is reflected on the entrance ports and therefore never enters the system.
    \item[$S^{io}$:] from outside to inside: signal that enters the system.
    \item[$S^{ii}$:] from inside to inside: transmissions and reflections internal to the system.
    \item[$S^{oi}$:] from inside to outside: signal that leaves the system.
\end{description}
Only~$a^o$ is known: this corresponds to the input to the system and we control it.
$a^i$,~$b^i$ and~$b^o$ are unknown.
$b^o$ is the result we are looking for.
$b^i$ and~$a^i$ are nice to have as they allow us to see what is happening inside the system.

\paragraph{Solving the internal transmissions and reflections.}
How are~$a^i$ and~$b^i$ related?
\Cref{eq:coupled_inputs_outputs} indicates that each element of $a^i$ is equal to an element of $b^i$, therefore $a^i$ can be obtain by reordering $b^i$ and vice-versa.
\index{permutation matrix}In other words, there exists a permutation matrix~$P$ such that
\begin{equation}
    a^i = P b^i \text{.} \label{eq:relation_ai_bi}
\end{equation}

To get $b_o$, the first step is to compute~$b_i$ from~$a_o$.
\begin{subequations}
    \begin{align}
        b^i &= S^{io}a^o + S^{ii}a^i \label{eq:compute_bi_ai} \\
        b^i &= S^{io}a^o + S^{ii}Pb^i \label{eq:compute_bi_bi} \\
        (I - S^{ii}P)b^i &= S^{io}a^o \label{eq:compute_bi_solve} \\
        b^i &= (I - S^{ii}P)^{-1} S^{io}a^o \label{eq:compute_bi_invert}
    \end{align} \label{eq:compute_bi}
\end{subequations}
where~$I$ is the identity matrix with a dimension equal to that of~$S^{ii}P$.

Can we go from~\cref{eq:compute_bi_solve} to~\cref{eq:compute_bi_invert}?
The matrix~$I - S^{ii}P$ has an inverse if and only if it is not singular.
A matrix is singular if and only if its determinant equals~0.
Can the determinant of~$I - S^{ii}P$ equal~0 (\cref{eq:det_i_minus_siip})?
\begin{equation}
    \det(I - S^{ii}P) = 0 \label{eq:det_i_minus_siip}
\end{equation}
\index{eigenvalue}\Cref{eq:det_i_minus_siip} has the shape of an eigenvalue problem;
\cref{eq:eigenvalue_typical} defines the eigenvalues~$\lambda$ of a matrix~$A$.
\begin{equation}
    \det(A - \lambda I) = 0 \label{eq:eigenvalue_typical}
\end{equation}
In our case, $\lambda=1$ and $A=S^{ii}P$.
However, the sign is not the same: $I-A \neq A-I$.
Fortunately, the determinant has the following property~\cref{eq:determinant_scalar_multiplication}:
\begin{equation}
    \det(c A) = c^n \det(A) \label{eq:determinant_scalar_multiplication}
\end{equation}
for any $n$--by--$n$ matrix $A$.
Therefore,
\begin{subequations}
    \begin{align}
        \det(I - S^{ii}P)
        &= \det(-1(S^{ii}P - I)) \\
        &= (-1)^n \det(S^{ii}P - I) \text{.}
    \end{align}
\end{subequations}
We do not need to worry about the parity of~$n$.
Indeed, if $x=0$ then $(-1)^n x = 0$ as well.
This means that for our eigenvalue problem, the sign does not matter, as summarized by \cref{eq:determinant_sign_does_not_matter}.
\begin{equation}
    \det(I - S^{ii}P) = 0
    \quad
    \Longleftrightarrow
    \quad
    \det(S^{ii}P - I) = 0
    \label{eq:determinant_sign_does_not_matter}
\end{equation}
\index{eigenvector}We have succesfully identified our question of the invertibility of $I - S^{ii}P$ with a question regarding the eigenvectors of $S^{ii}P$ for the eivenvalue 1.
Saying $\det(S^{ii}P - I) = 0$ is equivalent to saying that there exist non-zero vectors $v$, called ``eigenvectors'', such that $v = S^{ii}Pv$.
The following argument, based on physical considerations, will demonstrate that there are no such vectors.

$b^i$ is what $S^{ii}P$ is applied to in \cref{eq:compute_bi}, therefore we are looking for vectors $b^i$ satisfying
\begin{equation}
    b^i = S^{ii}P b^i \text{.} \label{eq:bi_eigenvector}
\end{equation}
The two equations~\cref{eq:bi_eigenvector} and ~\cref{eq:compute_bi_bi}, taken together, 
lead to the following interesting equality:
\begin{equation}
    \left\lbrace
        \begin{aligned}
            b^i &= S^{ii}P b^i \\
            b^i &= S^{io}a^0 + S^{ii}P b^i
        \end{aligned}
    \right.
    \quad
    \Longleftrightarrow
    \quad
    S^{io}a^o = 0
\end{equation}
The only way for $S^{io}a^o$ to equal 0 for any value of $a^o$ is for $S^{io}$ to contain only zeros.
In other words, the system is totally blind: no energy from the outside world ($a^o$) ever enters the system.
Blind systems cannot be solved by our method but this is not a problem since there is no point in trying to build them in the first place.
For the rest of this demonstration, let us assume that the system is not blind, $S^{io}\neq 0$, but that instead we merely set $a^o$ to 0; let us also assume $b^i \neq 0$ (we somehow managed to inject energy inside the system before it got blind).
What would $b^i = S^{ii}P b^i$ mean?

$b^i$ is the list of all the outputs of the inside part of the system, fed back as input to that same system.
What \cref{eq:bi_eigenvector} means is that there exists some fields that are unaffected by the system; if such a field would enter the system, it would be trapped inside the system, looping through $S^{ii}P$ forever.
If $S^{ii}P$ has an eigenvector, then our system is totally lossless (for some fields) and can store any amount of energy forever.
We know that all our networks have losses; even the propagation in free space has losses in the form of the beam diffracting slowly and becoming wider than its target, essentially leaking energy.
Without using active components (external energy source, amplifiers), the best we can do is to aim for very low losses, which would produce very high-Q cavities but not perpetual storages.
Because our systems have losses, $S^{ii}P$ has no eigenvectors.
Therefore $S^{ii}P$ has no eigenvalues.
Therefore 1 is not an eigenvalue.
Therefore $\det(I-S^{ii}P) \neq 0$.
Therefore $I-S^{ii}P$ has an inverse.
Therefore \cref{eq:compute_bi_invert} is correct, and the system can be solved.

It can be interesting to draw a parallel between our problem and the convergence of an infinite series.
Indeed, the sum of an infinite series of ratio $q$ converges if $\abs{q} < 1$:
\begin{equation}
    \sum_{i=0}^\infty q^i = \frac{1}{1-q} = (1-q)^{-1}
\end{equation}
The ressemblance with $(I - S^{ii}P)^{-1}$ is not due to chance.
Our system acts like a cavity, maybe a very complex cavity but a cavity nontheless.
Therefore we can think about it as a simple cavity delimited by two semi-transparent mirrors.
The signal trapped inside the cavity is reflected an infinite amount of times, each double-reflection attenuates the signal by a factor $q=S^{ii}P$.
As a result, the signal inside the cavity is the signal that enters the cavity multiplied by $1+q+q^2+q^3+\dots$ up to infinity.
If there are losses, then $|q|<1$ and that series converges to $(1-q)^{-1}$ or, considering we are working with matrices, $I - S^{ii}P$.

Note that for the implementation, we may want to avoid inverting the matrix.
Indeed, inverting matrices is costly and numerically unstable \todo{citation needed}.
Even though \cref{eq:compute_bi_invert} is mathematically correct, we would rather solve $b^i$ in \cref{eq:compute_bi_solve}.

\paragraph{Solving the output of the system.}
Once we know the internal reflections of the system ($b^i$ as a function of itself), getting its output is trival.
The second step is to compute~$b^o$ from~$a^o$ and~$b^i$.
We do that by taking $b^o$ from \cref{eq:s_outside_inside} and eliminating $a^i$ with \cref{eq:relation_ai_bi}.
\begin{align}
    b^o &= S^{oo}a^o + S^{oi}a^i \\
    b^o &= S^{oo}a^o + S^{oi}Pb^i \label{eq:compute_bo}
\end{align}

These two steps solve the system: from the inputs~$a^o$ we can compute the outputs~$b^o$ of the whole system.
This method also tells gives us $a^i$ and $b^i$ which are two equivalent ways of looking at what is happening inside the system (they are permutations of each other).

\paragraph{Internal sources.}
The method described above does not allow us to account for sources inside the system.
Such sources may nevertheless exist.
For example, the local oscillator of HIFI is located outside the cryostat (\cref{fig:internal_sources_windows}).
Therefore, the local oscillator is warmer, radiating heat.
With the current state of our model, we can make this local oscillator power an external source.
However, the windows that physically separate the cryostat from the outside world are also relatively warm, and they also act as sources of radiation.

\begin{figure}[hbtp]
    \centering
    \missingfigure{Cryostat windows radiating power.}
    \caption{\label{fig:internal_sources_windows} The cryostat windows are warm enough to emit a significant power.}
\end{figure}

Let us assume that every port radiates a constant flux described by \cref{eq:internal_sources}.
The output $b_i$ of a port is a linear combination of the inputs $a_j$, plus a constant $c_i$.
\begin{equation}
    b = S a + c
    \label{eq:internal_sources}
\end{equation}
The separation into inside and outside ports is straight forward, as~\cref{eq:internal_sources_io} indicates.
\begin{equation}
    \begin{pmatrix}
        b^o\\
        b^i\\
    \end{pmatrix}
    =
    \begin{pmatrix}
        S^{oo} & S^{oi} \\
        S^{io} & S^{ii}
    \end{pmatrix}
    \begin{pmatrix}
        a^o\\
        a^i\\
    \end{pmatrix}
    +
    \begin{pmatrix}
        c^o\\
        c^i\\
    \end{pmatrix}
    \label{eq:internal_sources_io}
\end{equation}
Like previously, we solve for the internal transmissions and reflections.
\begin{subequations}
    \begin{align}
        b^i &= S^{io} a^o + S^{ii} a^i + c^i \label{eq:compute_bi_ai_ci}\\
        b^i &= S^{io} a^o + S^{ii} Pb^i + c^i \label{eq:compute_bi_bi_ci}\\
        (I - S^{ii} P) b^i &= S^{io} a^o + c^i \label{eq:compute_bi_ci_solve}
    \end{align}
    \label{eq:compute_bi_ci}
\end{subequations}
\Cref{eq:compute_bi_ci_solve} is very similar to~\cref{eq:compute_bi_solve} and is as easy to solve for~$b^i$.
The previous discussion regarding the invertibility of $(I - S^{ii} P)$, through the meaning of~\cref{eq:bi_eigenvector}, still applies here.

Once $b_i$ is known, we inject its value into the expression of $b^o$.
\begin{subequations}
    \begin{align}
        b^o &= S^{oo} a^o + S^{oi} a^i + c^o \label{eq:compute_bo_ai_co}\\
        b^o &= S^{oo} a^o + S^{oi} Pb^i + c^o \label{eq:compute_bo_bi_co}
    \end{align}
    \label{eq:compute_bo_co}
\end{subequations}
Equations~\eqref{eq:compute_bi_ci_solve} and~\eqref{eq:compute_bo_bi_co} give us the response $b$ of the system $(S, P)$ to the inputs $a$ and $c$.

%-----------------------------------------------------------------------------
\paragraph{Solving for each source independantly.}
When solving a system with multiple sources, it may be tempting to set all the inputs at the same time.
That is, filling $a^o$ and $c$, then computing $b$.
This is, in most cases, a mistake.
If two inputs are set at the same time, then they act as if they are phase-locked to each other, and it introduces interferences that cannot be produced with either input source off.
Unless the sources are actually phase-locked to each other, the proper way of calculating the response of the system is to solve it for each source individually.

In the case of HIFI, the radiation for the local oscillator, the cryostat windows, the black bodies and the sky must be solved separately.

Then, like is the case for any incoherent signals, the results must be converted from amplitude to power then added together.

%-----------------------------------------------------------------------------
\subsubsection{Implementation}
Although the mathematics are quite simple, the implementation requires careful book keeping of the many indices of the many matrices.

\begin{itemize}
    \item 
The system contains $N$ networks.
    \item 
Each network is modeled with a~$n_i$--by--$n_i$ scattering matrix, where~$n_i$ is the number of ports the network $i$, with $i \in \llbracket 1, N \rrbracket$.
    \item 
The whole system is modeled with a~$n$--by--$n$ scattering matrix, where~$n$ is the number of ports of the whole system, with $n = \sum_{i=1}^N n_i$.
\end{itemize}

Each network $i$ comes with ports that are ordered from 1 to $n_i$.
The networks themselves are ordered since they are indexed from 1 to $N$.
Therefore, there is a natural way of numbering the ports of the whole system.
The port~$j$ of the network~$i$ corresponds to the port~$k$ of the whole system, where
\begin{equation}
    k = \sum_{m=0}^{i - 1}n_m + j \label{eq:port_numbering}
\end{equation}

With that information, it is trivial to fill the scattering matrix of the whole system, $S$, with values from the scattering matrices from each network $S_i$.
This is what \cref{algo:gather_networks} does.
\begin{algorithm}
    \caption{GatherNetworks}
    \label{algo:gather_networks}
    \begin{algorithmic}
        \Require{$N$} \Comment{Number of networks.}
        \Require{$(S_1, S_2, \ldots, S_N)$} \Comment{Scattering matrices of each network.}
        \Function{GatherNetworks}{$N$, $(S_1, S_2, \ldots, S_N)$}
        \State{}\Comment{Count the total number of ports.}
        \State {$n \gets 0$}
        \For{$i=1$ to $n$}
            \State{$n \gets n + $\Call{Size}{$S_i$}} \Comment{size = nb rows = nb columns: square matrices.}   
        \EndFor
        \State{}\Comment{Create the $n$-ports system scattering matrix.}
        \State{$S \gets n$--by--$n$ matrix filled with 0}
        \State{}\Comment{Fill its diagonal with the network scattering matrices.}
        \State{$\textit{start} \gets 1$} \Comment{Fill from row 1 column 1.}
        \For{$i=1$ to $n$}
            \State{$n_i \gets $\Call{Size}{$S_i$}}
            \State{$\textit{stop} \gets \textit{start} + n_i - 1$}
            \State{$S_{\textit{start}:\textit{stop}, \textit{start}:\textit{stop}} \gets S_i$}
            \Comment{Copy $S_i$ into $S$.}
            \State{$\textit{start} \gets \textit{start} + n_i$}
        \EndFor
        \\ \Return{$S$}
        \EndFunction
    \end{algorithmic}
\end{algorithm}
$S$ contains mostly zeros, except for blocks on its diagonal containing copies of each~$S_i$.
For example, \cref{eq:block_scattering} shows what $S$ looks like for a system containing two networks $S_1$ and $S_2$.
\begin{equation}
    S =
    \begin{pmatrix}
        {S_1}_{1, 1} & \cdots & {S_1}_{1, n_1} &
        0 & \cdots & 0
        \\
        \vdots & \ddots &\vdots &
        \vdots & \ddots &\vdots 
        \\
        {S_1}_{n_1, 1} & \cdots & {S_1}_{n_1, n_1} &
        0 & \cdots & 0
        \\
        0 & \cdots & 0 &
        {S_2}_{1, 1} & \cdots & {S_2}_{1, n_2}
        \\
        \vdots & \ddots &\vdots &
        \vdots & \ddots &\vdots
        \\
        0 & \cdots & 0 &
        {S_2}_{n_2, 1} & \cdots & {S_2}_{n_2, n_2}
    \end{pmatrix}
    \label{eq:block_scattering}
\end{equation}

Note that the number of zeros increases with~$N^2$ while the number of non-zeros elements around the diagonal increases with~$N$.
This remark has no importance from a purely mathematical point of view but is very important for the implementation of the solving algorithm.
\index{sparse matrix}Indeed, matrices that contain many zeros are called ``sparse matrices'' and many algorithms exist that have been optimized for dealing with them.
If we use such an optimized algorithm, then the matrix inversion that appears in \cref{eq:compute_bi_invert} will execute in $O(N)$ instead of $O(N^2)$ (using the ``big O notation'').

Unless we are lucky, the system scattering matrix $S$ (that we build from the network scattering matrices $S_i$) is not organized into inside and outside sectors like shown on \cref{eq:s_outside_inside}.
\index{permutation matrix}However, there exists a permutation matrix~$Q$ such as
\begin{gather}
    \left\lbrace
    \begin{aligned}
        S' &= Q S Q^{-1} \\
        a' &= Q a \\
        b' &= Q b \\
        c' &= Q c
    \end{aligned}
    \right.
    \label{eq:permute_s}
    \\
    b = S a + c\quad \Longleftrightarrow \quad b' = S' a' +c'\label{eq:permute_s_equiv}
\end{gather}
with $S'$ having the form of \cref{eq:s_outside_inside}.

The permutation matrix $Q$ must not be confused with the permutation matrix $P$ from \cref{eq:relation_ai_bi}, they are distinct, and both are needed.
$Q$~separates the inside ports from the outside ports, while $P$~describes how the inside ports are coupled.
Once the permutation matrix~$Q$ is known, we can solve the problem for~$S'$ and get~$b'$, from which we can retrieve~$b$~\cref{eq:permute_s_equiv}.

Permutation matrices permute the order of rows and columns of matrices.
They contain one and only one 1 per row and per column, all the other elements are 0.
Pre-multiplying by a permutation matrix changes the order of the rows.
Post-multiplying by a permutation matrix changes the order of the columns.
The transpose of a permutation matrix is also a permutation matrix.
The transpose of a permutation matrix is also its inverse: $P\transp = P^{-1}$.
This last property is important for optimization purposes because inverting matrices is expensive and unstable.

Both permutation matrices~$P$ and~$Q$ can be derived from a single dataset.
That dataset is a description of the ports that are coupled.
That dataset obviously defines~$P$, the matrix that describe the couplings.
But it also defines~$Q$, the matrix that separates the inside ports from the outside ports.
Indeed, any port present in that dataset is by definition an inside port, and any port absent from that dataset is an outside port.

\index{coupling descriptor}One way of representing the coupling of two ports is with a set of cardinal~2 that I name ``coupling descriptor''.
Its two elements are the two identifiers of the ports that are coupled.
For example, if the port 4 of the system is coupled to the port 10, then we can represent that coupling with the set $\lbrace 4, 10\rbrace$ which is equal to the set $\lbrace 10, 4\rbrace$ (sets are unordered).
Enforcing a cardinal of 2 prevents the coupling of a port to itself: $\lbrace 4, 4\rbrace = \lbrace 4\rbrace$ which has a cardinal of~1.
To be valid, each element must be between 1 and $n$, the number of ports in the whole system.

A set of coupling descriptors defines all the couplings that exist in the system.
To be valid, each of the coupling descriptors that it contains must be valid, and each port identifier can appear at most once.
For example,
$\lbrace \lbrace 1, 2 \rbrace, \lbrace 5, 3 \rbrace \rbrace$
is valid, but 
$\lbrace \lbrace 1, 2 \rbrace, \lbrace 5, 2 \rbrace \rbrace$
is not because the port identifier~2 appears more than once.

There are many possible permutation matrices $Q$: all that is required for $Q$ is to separate the inside ports from the outside ports, there is no constraint on the order of the ports beyond that.
To ensure the reproducibility of our algorithm, we want to choose one specific $Q$, the one for which the inside and outside port identifiers are in ascending order.
This is what the \cref{algo:separate_inside_outside} provides.
\Cref{algo:separate_inside_outside} uses the outputs of algorithms~\ref{algo:find_inside} and~\ref{algo:find_outside}.

\begin{algorithm}
    \caption{FindInside}
    \label{algo:find_inside}
    \begin{algorithmic}
        \Require {\textit{couplings}} \Comment{a valid set of coupling descriptors}
        \Function{FindInside}{\textit{couplings}}
        \State {\textit{inside} $\gets \emptyset$}
        \ForAll{\textit{coupling} $\in$ \textit{couplings}}
            \State {\textit{inside} $\gets$ \textit{inside} $\cup$ \textit{coupling}}
        \EndFor
        \\ \Return {\textit{inside}} \Comment{a set containing all the coupled ports}
        \EndFunction
    \end{algorithmic}
\end{algorithm}

\begin{algorithm}
    \caption{FindOutside}
    \label{algo:find_outside}
    \begin{algorithmic}
        \Require {\textit{inside}} \Comment{a set containing all the coupled ports}
        \Require {$n$} \Comment{number of ports in the system}
        \Function{FindOutside}{\textit{inside}, $n$}
        \State{\textit{allports} $\gets \lbrace 1, 2, 3, \dots, n \rbrace$}
        \State{\textit{outside} $\gets$ \textit{allports} $\setminus$ \textit{inside}}
        \\ \Return {\textit{outside}} \Comment{a set containing all the non-coupled ports}
        \EndFunction
    \end{algorithmic}
\end{algorithm}

\begin{algorithm}
    \caption{SeparateInsideOutside}
    \label{algo:separate_inside_outside}
    \begin{algorithmic}
        \Require {\textit{inside}} \Comment{a set containing all the coupled ports}
        \Require {\textit{outside}} \Comment{a set containing all the non-coupled ports}
        \Require {$n$} \Comment{number of ports in the system}
        \Function{SeparateInsideOutside}{\textit{inside}, \textit{outside}, $n$}
        \State{\textit{insorted} $\gets$ \Call{Sort}{\textit{inside}}}
        \State{\textit{outsorted} $\gets$ \Call{Sort}{\textit{outside}}}
        \State {$Q \gets$ $n$--by--$n$ matrix filled with 0}
        \For {$i = 1$ to \Call{Length}{\textit{outsorted}}}
            \State{$Q_{i, \textit{outsorted}[i]} \gets$ 1}
        \EndFor
        \For {$i$ = 1 to \Call{Length}{\textit{insorted}}}
            \State{$Q_{i + \Call{Length}{\textit{outsorted}}, \textit{insorted}[i]}$ $\gets$ 1}
        \EndFor
        \\ \Return {$Q$} \Comment{Permutation matrix that separates inside/outside ports}
        \EndFunction
    \end{algorithmic}
\end{algorithm}

Example: A system with $n=4$ ports has its ports 1 and 3 coupled together.
\Cref{algo:separate_inside_outside_example} shows how to compute $Q$ from a set of coupling descriptors.
The ``Ensure'' statements are assertions, propositions that must be true if the algorithm is working properly.

\begin{algorithm}[H]
    \caption{SeparateInsideOutside, example}
    \label{algo:separate_inside_outside_example}
    \begin{algorithmic}
        \State {\textit{couplings} $\gets \lbrace \lbrace 3, 1 \rbrace \rbrace$}
        \State {$n \gets 4$}
        \State {\textit{inside} $\gets$ \Call{FindInside}{\textit{couplings}}}
        \Ensure {\textit{inside} = $\lbrace 3, 1 \rbrace$} \Comment {Order does not matter for set equality.}
        \State {\textit{outside} $\gets $ \Call{FindOutside}{\textit{inside}, $n$}}
        \Ensure {\textit{outside} $= \lbrace 2, 4 \rbrace$}
        \State {$Q \gets$ \Call{SeparateInsideOutside}{\textit{inside}, \textit{outside}, $n$}}
        \Ensure {
        $
        Q = \begin{pmatrix}
            0 & 1 & 0 & 0\\
            0 & 0 & 0 & 1\\
            1 & 0 & 0 & 0\\
            0 & 0 & 1 & 0
        \end{pmatrix}
        $
        }
    \end{algorithmic}
\end{algorithm}
Let us apply the permutation matrix $Q$ produced in the example \cref{algo:separate_inside_outside_example} to a vector~$a$ and a scattering matrix~$S$, following \cref{eq:permute_s}.
The results are given in equations~\eqref{eq:q_a_aprime} and~\eqref{eq:q_s_qtranspose_sprime}.
\begin{equation}
    Q a
    =
    \begin{pmatrix}
        0 & 1 & 0 & 0\\
        0 & 0 & 0 & 1\\
        1 & 0 & 0 & 0\\
        0 & 0 & 1 & 0
    \end{pmatrix}
    \begin{pmatrix}
        a_1 \\ a_2 \\ a_3 \\ a_4
    \end{pmatrix}
    =
    \begin{pmatrix}
        a_2 \\ a_4 \\ a_1 \\ a_3
    \end{pmatrix}
    =
    a'\text{.}
    \label{eq:q_a_aprime}
\end{equation}
\begin{equation}
    Q S Q^{-1}
    =
    \begin{pmatrix}
        S_{2,2} & S_{2,4} & S_{2,1} & S_{2,3} \\
        S_{4,2} & S_{4,4} & S_{4,1} & S_{4,3} \\
        S_{1,2} & S_{1,4} & S_{1,1} & S_{1,3} \\
        S_{3,2} & S_{3,4} & S_{3,1} & S_{3,3}
    \end{pmatrix}
    =
    S'
    \label{eq:q_s_qtranspose_sprime}
\end{equation}
The next step, separating $a'$ in two and $S'$ in four, is trivial to implement.
\Cref{eq:q_s_qtranspose_sprime_decomposed} shows a decomposition of the matrix $S'$ from \cref{eq:q_s_qtranspose_sprime} into inside and outside submatrices.
\begin{equation}
    \begin{aligned}
    S'^{oo}
    &=
    \begin{pmatrix}
        S_{2,2} & S_{2,4}  \\
        S_{4,2} & S_{4,4}  \\
    \end{pmatrix}
    &
    S'^{oi}
    &=
    \begin{pmatrix}
        S_{2,1} & S_{2,3}  \\
        S_{4,1} & S_{4,3}  \\
    \end{pmatrix}
    \\
    S'^{io}
    &=
    \begin{pmatrix}
        S_{1,2} & S_{1,4}  \\
        S_{3,2} & S_{3,4}  \\
    \end{pmatrix}
    &
    S'^{ii}
    &=
    \begin{pmatrix}
        S_{1,1} & S_{1,3}  \\
        S_{3,1} & S_{3,3}  \\
    \end{pmatrix}
    \end{aligned}
    \label{eq:q_s_qtranspose_sprime_decomposed}
\end{equation}
Our algorithm successfully reorganized our data in a way compatible with \cref{eq:s_outside_inside}.

We now wish to construct the permutation matrix~$P$ that appears in \cref{eq:relation_ai_bi}.
Applied to our example, this corresponds to \cref{eq:relation_ai_bi_example}.
\begin{equation}
    \begin{pmatrix}
        a_1 \\ a_3
    \end{pmatrix}
    =
    P
    \begin{pmatrix}
        b_1 \\ b_3
    \end{pmatrix}
    \label{eq:relation_ai_bi_example}
\end{equation}
In our example, the ports 1 and 3 are coupled, therefore $a_1 = b_3$ and $a_3 = b_1$, which leads to \cref{eq:relation_bi_bi_example}.
\begin{equation}
    \begin{pmatrix}
        b_3 \\ b_1
    \end{pmatrix}
    =
    P
    \begin{pmatrix}
        b_1 \\ b_3
    \end{pmatrix}
    \label{eq:relation_bi_bi_example}
\end{equation}
It is obvious that the correct value for $P$ is that of \cref{eq:p_example}, but we need an algorithm to compute~$P$ for us in the general case.
\begin{equation}
    P =
    \begin{pmatrix}
        0 & 1 \\
        1 & 0
    \end{pmatrix}
    \label{eq:p_example}
\end{equation}
One difficulty comes from the fact that we cannot use the indices of the coupled ports directly to access the matrices $a'^i$, $b'^i$ and $S'^{ii}$.
Indeed, in our example, $b^i$ contains two elements only so we cannot reach $b_3$ by using directly the index 3.
We need to map the indices given by the set of coupling descriptors to indices that can be used to address the various matrices.
In our example, 1 is mapped to 1 ($b'^i_1 = b_1$), and 3 is mapped to 2 ($b'^i_2 = b_3$); therefore saying that ports 1 and 3 are coupled is equivalent to saying that indices 1 and 2 need to be permuted.
\Cref{algo:port_to_index} converts a port identifier to an index usable to address $S'^{ii}$.
\begin{algorithm}
    \caption{PortToIndex}
    \label{algo:port_to_index}
    \begin{algorithmic}
        \Require{\textit{port}} \Comment{The port identifier for which we want the index.}
        \Require{\textit{inside}} \Comment{Set of all the inside port identifiers.}
        \Function{PortToIndex}{\textit{port}, \textit{inside}}
        \State{\textit{insorted} $\gets$ \Call{Sort}{\textit{inside}}}
        \State{\textit{index} $\gets$ position of \textit{port} in \textit{insorted}}
        \\ \Return{\textit{index}}
        \EndFunction
    \end{algorithmic}
\end{algorithm}

We can now construct the permutation matrix $P$ with \cref{algo:couple_inputs_to_outputs}.
\begin{algorithm}
    \caption{CoupleInputsToOutputs}
    \label{algo:couple_inputs_to_outputs}
    \begin{algorithmic}
        \Require{\textit{inside}} \Comment{Set containing all the coupled ports.}
        \Require{\textit{couplings}} \Comment{Set of coupling descriptors.}
        \Function{CoupleInputsToOutputs}{\textit{couplings}}
        \State {$m \gets$ \Call{cardinal}{\textit{inside}}}
        \State {$P \gets$ $m$--by--$m$ matrix filled zith 0}
        \ForAll {\textit{coupling} $\in$ \textit{couplings}}
            \State {\textit{couplingArray} $\gets$ \Call{SetToArray}{\textit{coupling}}}
            \State {\textit{portA} $\gets$ \textit{couplingArray}[1]}
            \State {\textit{portB} $\gets$ \textit{couplingArray}[2]}
            \State {\textit{indexA} $\gets$ \Call{PortToIndex}{\textit{portA}, \textit{inside}}}
            \State {\textit{indexB} $\gets$ \Call{PortToIndex}{\textit{portB}, \textit{inside}}}
            \State {$P_{\textit{indexA}, \textit{indexB}} \gets$ 1}
            \State {$P_{\textit{indexB}, \textit{indexA}} \gets$ 1}
        \EndFor
        \\ \Return {$P$}
        \EndFunction
    \end{algorithmic}
\end{algorithm}

If we apply \cref{algo:couple_inputs_to_outputs} to our example, then there is only one value for \textit{coupling}: $\lbrace 3, 1\rbrace$.
Therefore, \textit{couplingArray}=[3, 1], \textit{portA}=3, \textit{portB}=1.
The two calls to PortToIndex take \textit{inside}=$\lbrace 3, 1\rbrace$ and make a sorted array from it, \textit{insorted}=[1, 3].
The position of \textit{portA} in that sorted array is 2, that of \textit{portB} is 1, so \textit{indexA}=2 and \textit{indexB}=1.
The matrix $P$ is filled with zeros, except for the elements $P_{1, 2}$ and $P_{2, 1}$ that equal 1, matching \cref{eq:p_example}.

\Cref{algo:solve_couplings} combines algorithms~%
\ref{algo:find_inside},
\ref{algo:find_outside},
\ref{algo:separate_inside_outside} and
\ref{algo:couple_inputs_to_outputs}
to produce $P$ and $Q$ from the set of coupling descriptors and the total number of ports.
\begin{algorithm}
    \caption{SolveCouplings}
    \label{algo:solve_couplings}
    \begin{algorithmic}
        \Require{\textit{couplings}} \Comment{Set of coupling descriptors.}
        \Require{$n$} \Comment{Number of ports in the whole system.}
        \Function{SolveCouplings}{\textit{couplings}, $n$}
        \State{$\textit{inside} \gets \Call{FindInside}{\textit{couplings}}$}
        \State{$\textit{outside} \gets \Call{FindOutside}{\textit{inside}, n}$}
        \State{$Q \gets \Call{SeparateInsideOutside}{\textit{inside}, \textit{outside}, n}$}
        \State{$P \gets \Call{CoupleInputsToOutputs}{\textit{couplings}}$}
        \State{$n^o \gets \Call{Cardinal}{\textit{outside}}$} \Comment{Number of outside ports.}
        \\ \Return $P, Q, n^o$
        \EndFunction
    \end{algorithmic}
\end{algorithm}

We have now everything we need to solve \cref{eq:compute_bi_ci_solve}, which I reproduce here (\cref{eq:compute_bi_solve_prime}) using the \textit{prime} version of the variables resulting from the permutations involving~$Q$.
\begin{equation}
    (I - S'^{ii})b'^i = S'^{io}a'^o + c'^i \label{eq:compute_bi_solve_prime}
\end{equation}
In \cref{eq:compute_bi_solve_prime}, everything is known except~$b'^i$.
We wish to solve this equation for~$b'^i$.
This is a very common problem for which many numerical packages offer solutions.
Solving systems of linear equations is also decribed at length in the chapter 2 of Numerical Recipes \cite{Press:2007:NRE:1403886}.
The developer is free to choose any method.
I suggest however something along the line of a LU decomposition.
Indeed, that LU decomposition can be computed once, its result stored, and then used several times to solve the same system for different values of $a'^o$ and $c'^i$.
Big systems may benefit from using sparse matrices.
\Cref{algo:solve_networks} separates the scattering matrix of the whole system into its four regions and facorizes $(I - S'^{ii})$.
\begin{algorithm}
    \caption{SolveNetworks}
    \label{algo:solve_networks}
    \begin{algorithmic}
        \Require{$P$} \Comment{Permutation matrix connecting inside inputs to outputs.}
        \Require{$Q$} \Comment{Permutation matrix separating inside from outside ports.}
        \Require{$n^o$} \Comment{Number of outside ports.}
        \Require{\textit{networks}} \Comment{List of scattering matrices of each network}
        \Function{SolveNetworks}{$P, Q, n^o, \textit{networks}$}
        \State{$N \gets \Call{Length}{\textit{networks}}$} \Comment{Number of networks.}
        \State{$S \gets \Call{GatherNetworks}{N, \textit{networks}}$}
        \State{$S' \gets Q S Q\transp$} \Comment{Outside ports are now at the beginning.}
        \State{}\Comment{Split the system scattering matrix into its four inside/outside regions.}
        \State{$n \gets \Call{Size}{S'}$}
        \State{$S'^{oo} \gets S'_{1:n^o, 1:n^o}$}
        \State{$S'^{oi} \gets S'_{1:n^o, n^o+1:n}$}
        \State{$S'^{io} \gets S'_{n^o+1:n, 1:n^o}$}
        \State{$S'^{ii} \gets S'_{n^o+1:n, n^o+1:n}$}
        \State{} \Comment{Part of the solving that is independant from the inputs.}
        \State{$I \gets \text{Identity matrix of size } S'^{ii} P$}
        \State{$\textit{LUdata} = \Call{LUdecompose}{I - S'^{ii} P}$}
        \\ \Return{$S'^{oo}$, $S'^{oi}$, $S'^{io}$, \textit{LUdata}}
        \EndFunction
    \end{algorithmic}
\end{algorithm}

Finally, the last step consists in applying the inputs to the system.
This is described by \cref{algo:solve_inputs}.
\begin{algorithm}
    \caption{SolveInputs}
    \label{algo:solve_inputs}
    \begin{algorithmic}
        \Require{$P$} \Comment{Permutation matrix connecting inside inputs to outputs.}
        \Require{$S'^{oo}, S'^{oi}, S'^{io}$}
        \Require{\textit{LUdata}} \Comment{Some factorization of $I - S'^{ii} P$.}
        \Require{$a'^o$} \Comment{Inputs entering the system.}
        \Require{$c'^o$} \Comment{Constant radiation from outside ports.}
        \Require{$c'^i$} \Comment{Constant radiation from inside ports.}
        \Function{SolveInputs}{$S'^{oo}, S'^{oi}, S'^{io}, \textit{LUdata}, a'^o, c'^o, c'^i$}
        \State{$b'^i = \Call{LUsolve}{\textit{LUdata}, S'^{io} a'^o + c'^i}$} \Comment{\Cref{eq:compute_bi_ci_solve}}
        \State{$b'^o = S'^{oo} a'^o + S'^{oi} P b'^i + c'^o$} \Comment{\Cref{eq:compute_bo_bi_co}}
        \State{$b' = \Call{Concatenate}{b'^o, b'^i}$}
        \\ \Return{$b'$}
        \EndFunction
    \end{algorithmic}
\end{algorithm}

\Cref{algo:solve_system} puts all the pieces together.
This is an example.
In a real application, where we need to loop over frequencies and apply different inputs for each frequency, we would call SolveCouplings once for the whole system, then SolveNetworks for each set of parameters (such as the frequency), then call SolveInputs for each source of radiation.
This is illustrated by \cref{algo:solve_system_realistic}.
\begin{algorithm}
    \caption{SolveSystem}
    \label{algo:solve_system}
    \begin{algorithmic}
        \Require{\textit{couplings}} \Comment{Set of coupling descriptors.}
        \Require{\textit{networks}} \Comment{Scattering matrix of each network.}
        \Require{$n$} \Comment{Number of ports in the system.}
        \Require{$a$} \Comment{Inputs to the system.}
        \Require{$c$} \Comment{Constant radiation from internal sources.}
        \Function{SolveSystem}{$\textit{couplings}, \textit{networks}, n, a, c$}
        \State{$P, Q, n^o \gets \Call{SolveCouplings}{\textit{couplings}, n}$}
        \State{$S'^{oo}, S'^{oi}, S'^{ii}, \textit{LUdata} \gets \Call{SolveNetworks}{\textit{networks}, P, Q, n^o}$}
        \State{$c' \gets Q c$}
        \State{$c'^o \gets c'_{1:n^o}$}
        \State{$c'^i \gets c'_{n^o+1:n}$}
        \State{$a' \gets Qa$}
        \State{$a'^o \gets a'_{1:n^o}$} \Comment{There is no use for $a'^i$ since it is defined by $b$.}
        \State{$b' \gets \Call{SolveInputs}{a'^o, c'^o, c'^i, S'^{oo}, S'^{oi}, S'^{ii}, \textit{LUdata}}$}
        \State {$b \gets Q\transp b'$}
        \\ \Return {$b$}
        \EndFunction
    \end{algorithmic}
\end{algorithm}

\begin{algorithm}
    \caption{SolveSystemRealistic}
    \label{algo:solve_system_realistic}
    \begin{algorithmic}
        \Require{\textit{couplings}} \Comment{Set of coupling descriptors.}
        \Require{$n$} \Comment{Number of ports in the system.}
        \Require{\textit{networkparamset}}
        \Require{\textit{sourcesconfiguration}} \Comment{List of values for $a$ and $c$.}
        \Function{SolveSystemRealistic}{\textit{couplings}, $n$, \textit{networkparamset}, \textit{sourcesconfiguration}}
        \State{$P, Q, n^o \gets \Call{SolveCouplings}{\textit{couplings}, n}$}
        \ForAll{$\textit{networkparamset} \in \textit{networkparamsets}$}
            \State{$\textit{networks} \gets \Call{ComputeNetworks}{\textit{networkparamset}}$}
            \State{$S'^{oo}, S'^{oi}, S'^{ii}, \textit{LUdata} \gets \Call{SolveNetworks}{\textit{networks}, P, Q, n^o}$}
            \ForAll{$a, c \in \textit{sourcesconfiguration}$}
                \State{$c' \gets Q c$}
                \State{$c'^o \gets c'_{1:n^o}$}
                \State{$c'^i \gets c'_{n^o+1:n}$}
                \State{$a' \gets Qa$}
                \State{$a'^o \gets a'_{1:n^o}$} \Comment{There is no use for $a'^i$ since it is defined by $b$.}
                \State{$b' \gets \Call{SolveInputs}{a'^o, c'^o, c'^i, S'^{oo}, S'^{oi}, S'^{ii}, \textit{LUdata}}$}
                \State {$b \gets Q\transp b'$}
                \State{Store $b$ somewhere}
            \EndFor
        \EndFor
    \\ \Return {One $b$ for each network param set and source configuration.}
    \EndFunction
    \end{algorithmic}
\end{algorithm}





%#############################################################################

\section{Generic networks}

Scattering matrix of typical networks, with parameters.

Some of these parameters may not be super physical, but are here to kinda absorb the imperfections and the mismatch when fitting.

For the grid I use Houde \cite{houde_2001} because there is nothing simpler, but for the other networks I keep it as simple as I can.
Grids being VERY complex to model, I might spend a lot of time and space on that one.

So, in the end, it's a list of scattering matrices.  It would be nice to augment each system with plots showing the effect of each parameter.



%=============================================================================

\subsection{Distance}
\label{sec:generic_networks_distance}
Let us assume an homogeneous isotropic propagation medium with an refractive index $n$.
$n$ can be complex, the imaginary part models the absorption.

\begin{figure}[hbtp]
    \centering
    \missingfigure{Propagation in a homogeneous medium.}
    \caption{\label{fig:net_distance} Propagation in a homogeneous medium.}
\end{figure}
\Cref{fig:net_distance} shows two points $z_1$ and $z_2$ along the path of a plane wave traveling through a medium of index $n$.
The line joining the two points is parallel to the direction of propagation of the wave.
The two points are separated by a distance $d$.
Then, at any time, the electric fields at the two points are linked by \cref{eq:net_distance}.
\begin{equation}
    e(z_2) = e(z_1) \exp
    \left(
        - i 2\pi d n f / c_0
    \right)
    \label{eq:net_distance}
\end{equation}

The argument of the exponential can be separated into a real and an imaginary part as shown with \cref{eq:absorption_phase}.
\begin{gather}
    \begin{aligned}
        - i 2\pi d n f / c_0
        &= - i 2\pi d \left(\Re(n) + i\Im(n)\right) f / c_0 \\
        &= 2\pi d f / c_0 \left(\Im(n) - i\Re(n) \right) \\
        &= a + i \phi
    \end{aligned}
    \label{eq:absorption_phase}
    \\
    \begin{aligned}
        a &= 2\pi d \Im(n) f / c_0   &   \phi &= -2\pi d \Re(n) f / c_0
    \end{aligned}
\end{gather}
The real part~$a$ constitutes an absorption factor while the imaginary part~$\phi$ constitutes a phase factor.
The imaginary part of the refractive index is small for dielectric and big for metals,
leading to a very strong absorption of the wave in metals.
The minus sign for the phase comes from the fact that the wave reaches $z_1$ before it reaches $z_2$: the field in $z_2$ is late relatively to the field in $z_1$.
Note that $\Re(n) f / c_0 = 1 / \lambda$, with $\lambda$ the wavelength of the wave in the medium.

A distance of homogeneous isotropic propagation medium constitutes a two-ports network (\cref{eq:s_2_ports}).
\begin{equation}
    S =
    \begin{pmatrix}
        S_{1, 1} & S_{1, 2} \\
        S_{2, 1} & S_{2, 2}
    \end{pmatrix}
    \label{eq:s_2_ports}
\end{equation}
There are no reflections on the ports (reflections occur at interfaces between materials of different refractive indices, we shall study these later).
The transmission is the same both ways, and because of the isotropy, all three components of the field are affected in the same way.
\begin{equation}
    \begin{aligned}
    S_{1, 1} = S_{2, 2} &=
    \begin{pmatrix}
        0 & 0 & 0 \\
        0 & 0 & 0 \\
        0 & 0 & 0
    \end{pmatrix}
    \\ 
    S_{1, 2} = S_{1, 2} &=
    \exp(- i 2\pi d n f / c_0)
    \begin{pmatrix}
        1 & 0 & 0 \\
        0 & 1 & 0 \\
        0 & 0 & 1
    \end{pmatrix}
    \end{aligned}
    \label{eq:scattering_distance}
\end{equation}
The input and output waves propagates in the $z$ direction, therefore their electric and magnetic fields are contained in the $(x, y)$ plane and have no $z$ component.
This means that the content of the third column of the scattering matrix of \cref{eq:scattering_distance} does not matter as it is always multiplied by~0.
Since these values are arbitrary, let us use convenient ones.
We choose to fill the last column with zeros and place a~1 on the diagonal. 
This has an advantage: we do not need to multiply by rotation matrices when the system rotates, as we demonstrated with \vref{eq:jones_rotation_identity}.
This is less work to do for the computer and this optimization has no drawback.


%=============================================================================

\subsection{Interface at normal incidence}
\label{sec:generic_networks_interface_at_normal_incidence}

\begin{figure}[hbtp]
    \centering
    \missingfigure{Interface at normal incidence}
    \caption{\label{fig:net_interface_normal}Interface at normal incidence.}
\end{figure}
An interface is an implicit surface defined by a change in refractive index.
Generally, interfaces both reflect and transmit light.
The amount of reflection and transmission depends on the refractive indices on the two sides of the interface.
One particular case is the case where both indices are equal; in this case the interface does not really exist and has no effect on the wave: no reflection and full transmission.

When the direction of propagation is normal to the surface, we are in a case of normal incidence.
In case of normal incidence, the interface is a two-ports network and its scattering matrix has the shape of \cref{eq:s_2_ports}.

\Crefrange{eq:fresnel_normal_r}{eq:fresnel_normal_t} are the Fresnel equations \eqref{eq:fresnel_rp}~to~\eqref{eq:fresnel_ts} rewritten for the case of normal incidence.
\begin{subequations}
    \begin{align}
        r &= \frac{n_i - n_t}{n_i + n_t} \label{eq:fresnel_normal_r}\\
        t &= \frac{2 n_i}{n_i + n_t} \label{eq:fresnel_normal_t}
    \end{align}
    \label{eq:fresnel_normal}
\end{subequations}
The $i$ and $t$ subscripts stand for ``incident'' and ``transmitted''.
The two materials, ``material 1'' and ``material 2'', have for refraction indices $n_1$ and $n_2$.
When when going from material 1 to material 2, $n_i = n_1$ and $n_t = n_2$.
When when going from material 2 to material 1, $n_i = n_2$ and $n_t = n_1$.
Therefore, the four elements of the scattering matrix are given by \cref{eq:s_interface_normal} in which $I_3$ is the 3--by--3 identity matrix.
\begin{subequations}
    \begin{align}
        I_3 &= \begin{pmatrix} 1&0&0\\0&1&0\\0&0&1 \end{pmatrix}
        \\
        S_{1, 1} &= \frac{n_1 - n_2}{n_1 + n_2} I_3
        \\
        S_{2, 2} &= \frac{n_2 - n_1}{n_2 + n_1} I_3
        \\
        S_{1, 2} &= \frac{2 n_2}{n_2 + n_1} I_3
        \\
        S_{2, 1} &= \frac{2 n_1}{n_1 + n_2} I_3
    \end{align}
    \label{eq:s_interface_normal}
\end{subequations}

%=============================================================================

\subsection{Interface at oblique incidence}
Interfaces at oblique incidence are more complex to model than interfaces at normal incidence because they need more parameters to describe.
In addition to the two refractive indices, we need the direction of propagation and the orientation of the interface in space.
This is required because the reflection and transmission on an oblique interface depend on the polarization of the wave seen by the surface.

\index{plane-of-incidence}As described by Hecht in \cite{hecht2002optics} (chapter 4), reflection and transmission depend on whether the field is contained in, or is normal to, the plane-of-incidence.
The plane-of-incidence is the plane that contains both the direction of propagation and the normal to the interface.
In case of normal incidence, the plane-of-incidence is undefined.
\begin{figure}[hbtp]
    \centering
    \missingfigure{Interface at oblique incidence}
    \caption{\label{fig:net_interface_oblique}Interface at oblique incidence.}
\end{figure}

\index{Fresnel equations}\Crefrange{eq:fresnel_rp}{eq:fresnel_ts} are the Fresnel equations corresponding to the field directions described in figure \cref{fig:fresnel_directions} for linear homogeneous isotropic dielectric materials.
\begin{subequations}
    \begin{align}
        r_\parallel & =
        \frac{n_i \cos \theta_t - n_t \cos \theta_i}{n_i \cos \theta_t + n_t \cos \theta_i}
        \label{eq:fresnel_rp}
        \\
        r_\perp & =
        \frac{n_i \cos \theta_i - n_t \cos \theta_t}{n_i \cos \theta_i + n_t \cos \theta_t}
        \label{eq:fresnel_rs}
        \\
        t_\parallel & =
        \frac {2 n_i \cos \theta_i}{n_i \cos \theta_t + n_t \cos \theta_i}
        \label{eq:fresnel_tp}
        \\
        t_\perp & =
        \frac {2 n_i \cos \theta_i}{n_i \cos \theta_i + n_t \cos \theta_t}
        \label{eq:fresnel_ts}
    \end{align}
    \label{eq:fresnel_oblique}
\end{subequations}
\Cref{eq:fresnel_rp} differs from that given by Hecht in~\cite{hecht2002optics}.
This comes from a different convention for the direction of the reflected field.
Our convention for the direction of the parallel reflected field makes the parallel and perpendicular results consistent when the angle of incidence is zero.

% In Hecht's chapter 4, when you try to get the equations to line up for the
% case of normal incidence, then the two transmissions have the same limit,
% but the two reflections disagree.  Why?  It should not be the case, should
% it?  Which one is the correct one, if any?  When the angle is 0, the
% incidence plane is not defined.  Therefore it is not possible to be parallel
% to it.  Then again, it is not really possible to be perpendicular to it
% either.
%
% Hecht is mentioning something in his section about the Fresnel equations: the
% choice of reference frame matters.  By choosing another reference frame for
% the perp case, we can flip the sign of $r_perp$ and leave the rest untouched.
% Actually, when I look at his figures 4.39 and 4.40, I am not surprised that
% his signs are messed up since his vectors do not overlap in the same way when
% the angles converge toward 0.  Indeed, Ei and Et overlap in his perp case,
% but not in his para case.  So if I fix this, I get a minus sign for
% $r_para$.

\begin{figure}[hbtp]
    \centering
    \input{\MyGraphicsPath fresnel_reference_frames.pdf_tex}
    \caption{\label{fig:fresnel_directions}Field directions chosen to express the Fresnel equations.}
    \caption*{
        $n$ is normal to the interface, $u$ is normal to the plane of incidence.
        The vectors~$p_i$ indicate the direction of the field parallel to the plane-of-incidence (plane of the page) for each port.  The vectors~$s_i$ do the same for the perpendicular direction.
    }
\end{figure}

Before we can apply the reflection and transmission coefficients given in~\cref{eq:fresnel_oblique}, we need to decompose the incident field into a parallel and a perpendicular component.
The incident field can be decomposed into a component that is parallel to the plane-of-incidence and one that is perpendicular to it.

We need notations.
I call $k$ the direction of propagation, $n$ the normal to the interface, and $u$ the normal to the plane-of-incidence (because $p$ and $i$ are already taken).

We get $e_\perp$ by projecting $e$ on the unit vector $u$ according to \cref{eq:paraperp_1}.
\begin{equation}
    e_\perp = u \times (u \cdot e) \label{eq:paraperp_1}
\end{equation}
In \cref{eq:paraperp_1}, $u \cdot e$~represents the inner product (dot product) of~$u$ by~$e$ and the symbol~``$\times$'' refers to the multiplication of a matrix (here~$u$) by a scalar (here~$u \cdot e$).
We wish to replace that scalar multiplication with a matrix multiplication.
To do that, we replace the scalar~$u \cdot e$ by the 1--by--1 matrix~$[u \cdot e]$ containing that scalar as shown in \cref{eq:paraperp_2}.
We do not use any symbol for the matrix multiplication, we simply juxtapose the matrices.
\begin{equation}
    e_\perp = u [u \cdot e] \label{eq:paraperp_2}
\end{equation}
The dimensions match: $u$~is a 3--by--1 matrix and $[u \cdot e]$~is a 1--by--1 matrix, resulting in~$e_\perp$ being a 3--by--1 matrix.
For the next step, we note that the inner product $u \cdot e$ is related to the outer product $u\transp e$ by \cref{eq:paraperp_3}.
\begin{equation}
    [u \cdot e] = u\transp e \label{eq:paraperp_3}
\end{equation}
Injecting \cref{eq:paraperp_3} into \cref{eq:paraperp_2} gives \cref{eq:paraperp_4}.
\begin{equation}
    e_\perp = u \left( u\transp e \right) \label{eq:paraperp_4}
\end{equation}
By associativity of the matrix multiplication we can transform \cref{eq:paraperp_4} into \cref{eq:para_perp_decomposition_field}.
Then, by identification, we can find our decomposition matrices.
In \cref{eq:para_perp_decomposition_matrix}, the matrices~$M_\perp$ and~$M_\parallel$ decompose a vector~$e$ into its perpendicular and parallel components~$e_\perp$ and~$e_\parallel$ on which we can apply the Fresnel equations~\eqref{eq:fresnel_oblique}.
\begin{gather}
    \begin{aligned}
        e_\perp &= \left(u u\transp \right) e
        \\
        e_\parallel &= e - e_\perp
    \end{aligned}
    \label{eq:para_perp_decomposition_field}
    \\
    \begin{aligned}
        M_\perp &= u u\transp   \\
        M_\parallel & = I - M_\perp
    \end{aligned}
    \label{eq:para_perp_decomposition_matrix}
\end{gather}

One more step is required: we need to compute $u$ from $n$ and $k$.
This step is very straightforward.
Since $u$ is normal to the plane-of-incidence, and since the plane-of-incidence contains $n$ and $k$, then $u$ is normal to both $n$ and $k$.
In other words, $u$ is collinear to the cross product of $n$ and $k$.
We know that the cross-product is antisymetric but in our case the order does not matter; indeed, the sign cancels itself out in $u u^T$.
\begin{equation}
    u = \frac{1}{\norm{n \times k}} n \times k
\end{equation}
The order will matter in the next part when we consider the geometry of the system.

We are now able to calculate the reflected and transmitted fields of an incident plane wave on an interface at oblique incidence.
However, the results are in a reference frame local to the reflected or transmitted wave, not a global reference frame shared by the three waves.
\Cref{fig:fresnel_rotations} shows how the reflected and transmitted reference frames are obtained by rotating the incident reference frame around~$u$, the normal to the plane-of-incidence.
\begin{figure}[hbtp]
    \centering
    \missingfigure{Fresnel rotations.}
    \caption{\label{fig:fresnel_rotations}Fresnel rotations.}
\end{figure}

The angle-of-incidence, noted $\theta_i$, is defined in \cref{eq:angle_of_incidence_definition} as the angle between the incident direction of propagation~$k_i$ and the normal to the surface~$n$.
\begin{equation}
    \theta_i \equiv \widehat{n k_i} \pmod{2\pi}
    \label{eq:angle_of_incidence_definition}
\end{equation}
\Cref{eq:dot_product_cos} shows a relation between the dot-product of two vectors and the angle between them.
\begin{equation}
    v \cdot w = \norm{v} \norm{w} \cos \widehat{vw}
    \label{eq:dot_product_cos}
\end{equation}
The dot-product~$v \cdot w$ can also be calculated by summing the component-wise product of $v$ and $w$ (\cref{eq:dot_product_elementwise}, which is a method that does not require knowing~$\widehat{vw}$.
\begin{equation}
    v \cdot w = \sum_i v_i w_i
    \label{eq:dot_product_elementwise}
\end{equation}
\Cref{eq:dot_product_cos} and~\cref{eq:dot_product_elementwise} form a system from which we can extract the angle~$\widehat{uv}$.
Its solution is given in \cref{eq:angle_from_dot_product}, provided that the angle is between 0 and $\pi$.
This is due to a limitation of the range of the $\arccos$ function.
\begin{equation}
    \left\lbrace
        \begin{aligned}
            \widehat{vw} &\in [0, \pi]
            \\
            \widehat{vw} &= \arccos
            \left(
                \frac{
                    \sum_i v_i w_i
                }{
                    \norm{v} \norm{w}
                }
            \right)
        \end{aligned}
    \right.
    \label{eq:angle_from_dot_product}
\end{equation}
The limitation~$\widehat{vw} \in [0, \pi]$ does not impede us.
Indeed, $\theta_i \in ]0, \pi/2[$.
\begin{itemize}
    \item 0 is excluded because this corresponds to the case of normal incidence which results in a two-ports network and not a four-ports network.  We have treated this particular case in \vref{sec:generic_networks_interface_at_normal_incidence}.
    \item $\pi/2$ is excluded because this corresponds to a direction of propagation parallel to the surface, the wave would therefore never hit the interface.
    \item Angles between $\pi/2$ and $\pi$ are excluded because the incident wave would be coming from the wrong side of the interface.
    \item Angles between 0 and $-\pi$ are excluded because the rotation axis~$u$ adapts its direction to make sure that the angle from $n$ to $k_i$ is always positive from the point of view of~$u$.
    \item The other angles fall into the previous categories because of the modulo~$2\pi$.
\end{itemize}
It is therefore safe to apply \cref{eq:angle_from_dot_product} to retreive~$\theta_i$ from $n$ and $k_i$.

The vectors~$n$ and $k_1$ and the refractive incides $n_a$ and $n_b$ are sufficient to determine the geometry of the whole network.
In other words, the orientation of one port constrains that of the three other ports.
Because each port can be seen as an incident, reflected or transmitted port, I shall stop using the notation $\theta_i$, $\theta_r$ and $\theta_t$;
instead, I shall use $\theta_a$ and $\theta_b$.
$\theta_a$ refers to the angle between the direction of propagation and the normal to the interface on the A side of the interface;
likewise, $\theta_b$ refers to the corresponding angle on the B side.
\index{Snell's law}The angles~$\theta_a$ and~$\theta_b$ are related to the refractive incides~$n_a$ and~$n_b$ of the propagation media A and B by Snell's Law~\eqref{eq:snell}.
\begin{gather}
    \Re(n_a) \sin \theta_a = \Re(n_b) \sin \theta_b
    \label{eq:snell}
    \\
    \left\lbrace
        \begin{aligned}
            \theta_b &\in [-\pi/2, \pi/2]
            \\
            \theta_b &= \arcsin
            \left(
                \frac{\Re(n_a)}{\Re(n_b)}
                \sin \theta_a
            \right)
        \end{aligned}
    \right.
    \label{eq:snell_thetab}
\end{gather}
Once again, the limited range of~$\theta_b$ does not impede us.

Upon reflection and refraction, the direction of the electric field rotates around $u$.
Since there is no connection between the port 1 and the port 4, and between the port 2 and the port 3, we are interested in eight angles only: $\theta_{1 \rightarrow 2}$, $\theta_{1 \rightarrow 3}$, $\theta_{2 \rightarrow 1}$, $\theta_{2 \rightarrow 4}$, $\theta_{3 \rightarrow 1}$, $\theta_{3 \rightarrow 4}$, $\theta_{4 \rightarrow 2}$ and $\theta_{4 \rightarrow 3}$.
The values for these angles are listed in equations~\cref{eq:interface_rotations}.
\begin{equation}
    \begin{aligned}
        \theta_{1 \rightarrow 2} &= \pi - 2\theta_a
        &
        \theta_{2 \rightarrow 1} &= -\pi + 2\theta_a
        \\
        \theta_{3 \rightarrow 4} &= \pi - 2\theta_b
        &
        \theta_{4 \rightarrow 3} &= -\pi + 2\theta_b
        \\
        \theta_{1 \rightarrow 3} &= \theta_b - \theta_a
        &
        \theta_{3 \rightarrow 1} &= \theta_a - \theta_b
        \\
        \theta_{2 \rightarrow 4} &= \theta_a - \theta_b
        &
        \theta_{4 \rightarrow 2} &= \theta_b - \theta_a
    \end{aligned}
    \label{eq:interface_rotations}
\end{equation}
To each angle $\theta$ of \cref{eq:interface_rotations} corresponds a rotation matrix $R$.
From an implementation point of view, computing the matrix corresponding to a rotation around an axis (here $u$) may be best done using a quaternion.
We do not need to enter into details, the two following steps are all we need to create the rotation matrix we want.
First, we construct the quaternion corresponding to a rotation of~$\theta$ around a vector $u=(u_x, u_y, y_z)$ with \cref{eq:quaternion_rotation_around_axis}.
\begin{equation}
    \begin{aligned}
        q &= w + xi + yj + zk
        \\
        q &= \cos \frac{\theta}{2}
           + u_x \sin \frac{\theta}{2} i
           + u_y \sin \frac{\theta}{2} j
           + u_z \sin \frac{\theta}{2} k
    \end{aligned}
    \label{eq:quaternion_rotation_around_axis}
\end{equation}
Then we convert that quaternion into a rotation matrix with \cref{eq:quaternion_to_rotation_matrix}.
\begin{equation}
    R =
    \begin{pmatrix}
        1 - 2y^2 - 2z^2   &   2xy - 2zw         &   2xz + 2yw \\
        2xy + 2zw         &   1 - 2x^2 - 2z^2   &   2yz - 2xw \\
        2xz - 2yw         &   2yz + 2xw         &   1 - 2x^2 - 2y^2
    \end{pmatrix}
    \label{eq:quaternion_to_rotation_matrix}
\end{equation}

We can finally put it all together.
\Cref{eq:interface_S} lists all the elements of the scattering matrix of an interface at normal indicence.
The matrices $R$ are rotation matrices produced by~\cref{eq:quaternion_to_rotation_matrix} for each angle~\eqref{eq:interface_rotations}.
The matrices $M_\parallel$ and $M_\perp$ are defined in~\cref{eq:para_perp_decomposition_matrix} and decompose the fields into their parallel and perpendicular components.
The various~$r$ and~$t$ refer to the Fresnel equations~\eqref{eq:fresnel_oblique}; the subscript~$a$ denoting incidence on the A side of the interface, and~$b$ to the B side of the interface.
\begin{equation}
    \begin{gathered}
    \begin{aligned}
        S_{1, 2} &= R^{2 \rightarrow 1} \left(
            r_{\parallel a} M_\parallel +
            r_{\perp a} M_\perp
        \right)
        &
        S_{2, 1} &= R^{1 \rightarrow 2} \left(
            r_{\parallel a} M_\parallel +
            r_{\perp a} M_\perp
        \right)
        \\
        S_{1, 3} &= R^{3 \rightarrow 1} \left(
            t_{\parallel b} M_\parallel +
            t_{\perp b} M_\perp
        \right)
        &
        S_{3, 1} &= R^{1 \rightarrow 3} \left(
            t_{\parallel a} M_\parallel +
            t_{\perp a} M_\perp
        \right)
        \\
        S_{2, 4} &= R^{4 \rightarrow 2} \left(
            t_{\parallel b} M_\parallel +
            t_{\perp b} M_\perp
        \right)
        &
        S_{4, 2} &= R^{2 \rightarrow 4} \left(
            t_{\parallel a} M_\parallel +
            t_{\perp a} M_\perp
        \right)
        \\
        S_{3, 4} &= R^{4 \rightarrow 3} \left(
            r_{\parallel b} M_\parallel +
            r_{\perp b} M_\perp
        \right)
        &
        S_{4, 3} &= R^{3 \rightarrow 4} \left(
            r_{\parallel b} M_\parallel +
            r_{\perp b} M_\perp
        \right)
        \\
    \end{aligned}
    \\
    S_{1, 1} = S_{1, 4} = S_{2, 2} = S_{2, 3} = S_{3, 2} = S_{3, 3} = S_{4, 1} = S_{4, 4} = 0
    \end{gathered}
    \label{eq:interface_S}
\end{equation}
This concludes our work on the scattering matrix of the interface between two linear homogeneous isotropic media.
However, for this network to find its place in a system, we also need to compute the directions of propagations $k_2$, $k_3$ and $k_4$: these may be required by other networks.

\begin{figure}[hbtp]
    \centering
    \missingfigure{fig:interface propagation rotation}
    \caption{\label{fig:interface_propagation_rotation}interface propagation rotation}
\end{figure}
Equations~\crefrange{eq:interface_propagation_rotation_k2}{eq:interface_propagation_rotation_k2} list the rotations matrices that transform~$k_1$ into $k_2$, $k_3$ and $k_4$, represented in \cref{fig:interface_propagation_rotation}.
\begin{subequations}
    \begin{align}
        k_2 &= R^{1 \rightarrow 2} k_1 \label{eq:interface_propagation_rotation_k2} \\
        k_3 &= R^{1 \rightarrow 3} k_1 \label{eq:interface_propagation_rotation_k3} \\
        k_4 &= R^{1 \rightarrow 4} k_1 \label{eq:interface_propagation_rotation_k4}
    \end{align}
    \label{eq:interface_propagation_rotation_ki}
\end{subequations}
We already know~$R^{1 \rightarrow 2}$ and~$R^{1 \rightarrow 3}$ (see \cref{eq:interface_rotations}) but we do not have~$R^{1 \rightarrow 4}$ yet.
The angle~$\theta_{1 \rightarrow 4}$ is easy to determine with \cref{fig:interface_propagation_rotation} and is given by \cref{eq:interface_rotation_1_to_4}.
We can use~\cref{eq:quaternion_rotation_around_axis} and~\cref{eq:quaternion_to_rotation_matrix} to derive~$R^{1 \rightarrow 4}$ from~$\theta_{1 \rightarrow 4}$.
\begin{equation}
    \theta_{1 \rightarrow 4} = -\theta_a - \theta_b
    \label{eq:interface_rotation_1_to_4}
\end{equation}
With that final step, we can now connect the network to other networks since we can provide them with their own~$k_1$.



%=============================================================================
\subsection{Thin film}
The thin film can be seen as its own network, or as two interfaces separated by a given distance.  Both should yield the same result.

Here, we treat the thin film as its own network.
This is convenient because the direction of propagation after the film is identical to that before the film.  The person using this network does not have to worry about what happens between the interfaces which is dealt with automatically.

\subsubsection{Thin film at normal incidence}

Two interfaces separate three regions of space of refractive indices~$n_1$, $n_2$ and~$n_3$ (in most cases, $n_1=n_3$).
Each interface of the film reflects and transmits radiation according to the Fresnel equations for normal incidence~\eqref{eq:fresnel_normal}.
\Cref{fig:thin_film_normal} defines the notations for the reflection and transmission coeffients between the three regions of space.
The propagation between the two interfaces introduces a factor $a$.
\Cref{fig:thin_film_normal_collapsed} is what results of our modeling effort: a single network with two reflections and two transmissions.

\begin{figure}[hbtp]
    \centering
    \input{\MyGraphicsPath thin_film_normal.pdf_tex}
    \caption{Thin film at normal incidence.}
    \label{fig:thin_film_normal}
\end{figure}
\begin{figure}[hbtp]
    \centering
    \input{\MyGraphicsPath thin_film_normal_collapsed.pdf_tex}
    \caption{Thin film at normal incidence, collapsed.}
    \label{fig:thin_film_normal_collapsed}
\end{figure}

\Crefrange{eq:thin_film_normal_0}{eq:thin_film_normal_2} walk us through the derivation of~$r_{1, 3}$, the reflection coefficient of the thin film seen from the region of refractive index~$n_1$.
\begin{align}
    r_{1,3}
    &= r_{1,2} + t_{1,2}
        \left(
            a r_{2,3} a +
            a r_{2,3} a r_{2,1} a r_{2,3} a +
            \cdots
        \right)
       t_{2,1}
    \label{eq:thin_film_normal_0}
    \\
    r_{1, 3}
    &=
    r_{1, 2} + t_{1, 2} t_{2, 1} a^2r_{2, 3}
        \sum_{i=0}^{\infty} (a^2r_{2,1}r_{2,3})
    \label{eq:thin_film_normal_1}
    \\
    r_{1, 3}
    &=
    r_{1, 2} + t_{1, 2} t_{2, 1} a^2 r_{2, 3}
    \frac{1}{1 - a^2 r_{2, 1} r_{2, 3}}
    \label{eq:thin_film_normal_2}
\end{align}
Likewise, we can derive $r_{3, 1}$, $t_{1, 3}$ and $t_{3, 1}$.
They are given by~\crefrange{eq:thin_film_normal_r31}{eq:thin_film_normal_t31},
with~\cref{eq:thin_film_normal_r13} being a mere reminder of~\cref{eq:thin_film_normal_2}.
\begin{subequations}
    \begin{align}
        r_{1, 3}
        &=
        r_{1, 2} + t_{1, 2} t_{2, 1} a^2 r_{2, 3}
        \frac{1}{1 - a^2 r_{2, 1} r_{2, 3}}
        \label{eq:thin_film_normal_r13}
        \\
        r_{3, 1}
        &=
        r_{3, 2} + t_{3, 2} t_{2, 3} a^2 r_{2, 1}
        \frac{1}{1 - a^2 r_{2, 3} r_{2, 1}}
        \label{eq:thin_film_normal_r31}
        \\
        t_{1, 3}
        &=
        t_{1,2} t_{2,3} a \frac{1}{1 - a^2 r_{2, 3} r_{2, 1}}
        \label{eq:thin_film_normal_t13}
        \\
        t_{3, 1}
        &=
        t_{3,2} t_{2,1} a \frac{1}{1 - a^2 r_{2, 1} r_{2, 3}}
        \label{eq:thin_film_normal_t31}
    \end{align}
\end{subequations}
The parameters $r_{1, 2}$, $t_{1, 2}$, $r_{3, 2}$, $t_{3, 2}$, $r_{2, 1}$, $t_{2, 1}$,
$r_{2, 3}$ and $t_{2, 3}$ are determined by the Fresnel equations for normal incidence~\eqref{eq:fresnel_normal}.
The parameter $a$ is determined by $\exp(-i 2 \pi d n f / c_0)$ according to~\cref{eq:net_distance}.
Note that in these four equations, the fraction is the same;
it is sufficient to compute its value once only.



There is one more factor to apply: a compensation for the space taken by the film.
As illustrated in~\cref{fig:thin_film_normal_compensation},
the film has a thickness $d_2$ and
its center is located at distances $d_1$ and $d_3$ from other reference points.
The actual length of the medium 1 is not $d_1$ but $d_1 - d_2/2$.
Likewise, the wave travels a distance $d_3 - d_2/2$ in the medium 3.
\begin{figure}[hbtp]
    \centering
    \input{\MyGraphicsPath thin_film_normal_compensation.pdf_tex}
    \caption{Thin film at normal incidence, space compensation.}
    \label{fig:thin_film_normal_compensation}
\end{figure}
One way of accounting for this without changing any other network is to add some negative space on each side of the film.
Let $a_1$ and $a_3$ be the effect of these negative spaces.
\Crefrange{eq:negative_space_3}{eq:negative_space_3} apply~\cref{eq:net_distance} to the refractive indices $n_1$ and $n_3$ for the negative distance $-d_2/2$.
\begin{subequations}
    \begin{align}
        a_1 &= \exp \Big(-i 2 \pi (-d_2/2) n_1 f / c_0 \Big) \label{eq:negative_space_1}
        \\
        a_3 &= \exp \Big(-i 2 \pi (-d_2/2) n_3 f / c_0 \Big) \label{eq:negative_space_3}
    \end{align}
    \label{eq:negative_space}
\end{subequations}
The reflection on the left side crosses the negative space $a_1$ twice, therefore $r_{1, 3}$ must be multiplied by $a_1^2$.
Likewise, $r_{3, 1}$ must be multiplied by $a_3^2$.
Both transmissions $t_{1, 3}$ and $t_{3, 1}$ go through $a_1$ and $a_3$, therefore they must be multiplied by $a_1 a_3$.
This is summarized with~\crefrange{eq:thin_film_normal_compensated_r13}{eq:thin_film_normal_compensated_t31}.
\begin{subequations}
    \begin{align}
        r'_{1, 3} &= a_1^2   \, r_{1, 3} \label{eq:thin_film_normal_compensated_r13} \\
        r'_{3, 1} &= a_3^2   \, r_{3, 1} \label{eq:thin_film_normal_compensated_r31} \\
        t'_{1, 3} &= a_1 a_3 \, t_{1, 3} \label{eq:thin_film_normal_compensated_t13} \\
        t'_{3, 1} &= a_1 a_3 \, t_{3, 1} \label{eq:thin_film_normal_compensated_t31}
    \end{align}
    \label{eq:thin_film_normal_compensated}
\end{subequations}

From there, building the scattering matrix of the thin film is straightforward.
If we name $I_3$ the 3--by--3 identity matrix,
then the Jones matrices ~\crefrange{eq:thin_film_normal_s11}{eq:thin_film_normal_s22}
are the elements of the scattering matrix.
\begin{subequations}
    \begin{align}
        S_{1, 1} &= r'_{1, 3} I_3 \label{eq:thin_film_normal_s11} \\
        S_{1, 2} &= t'_{3, 1} I_3 \label{eq:thin_film_normal_s12} \\
        S_{2, 1} &= t'_{1, 3} I_3 \label{eq:thin_film_normal_s21} \\
        S_{2, 2} &= r'_{3, 1} I_3 \label{eq:thin_film_normal_s22}
    \end{align}
    \label{eq:thin_film_normal_sij}
\end{subequations}
If necessary, each of these Jones matrices can be adapted to account for the orientation of the thin film;
see~\vref{sec:rotating_jones_matrices}.

\subsubsection{Thin film at oblique incidence}

Then, we run simulations to show that it performs exactly like the combination of three elementary networks.


%=============================================================================
\subsection{Wire grid polarizer}
In their paper from \citeyear{houde_2001} \citetitle{houde_2001}, \textcite{houde_2001} present a set of equations that approximate the electric field at any point near a wire grid polarizer.
The incident wave is assumed plane, the surrounding propagation medium is assumed homogeneous and isotropic, and the wires are assumed to be free floating (no dielectric substrate) and to have a cylindrical section.

The set of equations that interest us here are the equations numbered 23 to 35, 62 and 63 in \cite{houde_2001}.
\begin{align}
    K^x &= \frac{E_0}{F} \cdot \alpha' \frac{N_x}{\Delta_x}
    \\
    K^\theta &= (-j) \frac{E_0}{F} \cdot (\gamma' \beta - \beta' \gamma) \frac{N_\theta}{\Delta_\theta}
\end{align}
with
\begin{align}
    N_x
    &=
    1 - j \frac{Z_s}{Z_0} \frac{ka}{2}
    \\
    \Delta_x
    &=
    (1 - \alpha^2) S_1 - j \frac{Z_s}{Z_0} \sqrt{1 - \alpha^2}H_1^{(2)} (k'a)
    \\
    N_\theta
    &=
    1 + j \frac{Z_s}{Z_0} \frac{2}{ka}
    \\
    \Delta_\theta
    &=
    \sqrt{1 - \alpha^2} H_1^{(2)} (k'a) + j \frac{Z_s}{Z_0} (1 - \alpha^2) S_1
\end{align}
and
\begin{equation}
    S_1 = H_0^{(2)} (k'a) + 2
    \sum_{n=1}^\infty
    H_0^{(2)}(k'nd) \cos (k \beta nd)
    \text{.}
    \label{eq:infinite_hankel}
\end{equation}
The following equations use the previous definitions.
They describe the reflected and transmitted fields in three dimensions for a grid in the $xy$-plane, wires along~$x$.
\begin{align}
    R^x
    &=
    -\frac{F}{E_0}
    \frac{\lambda}{\pi d}
    \frac{1 - \alpha^2}{\gamma} K^x
    \label{eq:houde_Rx}
    \\
    R^y
    &=
    \phantom{-}
    \frac{F}{E_0}
    \frac{\lambda}{\pi d}
    \left[
        \frac{\alpha \beta}{\gamma} K^x
        -
        j \frac{ka}{2} K^\theta
    \right]
    \\
    R^z
    &=
    -\frac{F}{E_0}
    \frac{\lambda}{\pi d}
    \left[
       \alpha K^x
       +
       j \frac{\beta}{\gamma} \frac{ka}{2} K^\theta
    \right]
    \\
    T^x &= \alpha' + R^x
    \\
    T^y
    &=
    \beta' +
    \frac{F}{E_0}
    \frac{\lambda}{\pi d}
    \left[
        \frac{\alpha \beta}{\gamma} K^x + j \frac{ka}{2} K^\theta
    \right]
    \\
    T^z
    &=
    \gamma' +
    \frac{F}{E_0}
    \frac{\lambda}{\pi d}
    \left[
        \alpha K^x - j \frac{\beta}{\gamma} \frac{ka}{2} K^\theta
    \right]
\end{align}
Note that Houde calls these $R$ and $T$ ``reflection'' and ``transmission coefficient'', something they are not quite that since they contain $\alpha'$, $\beta'$ and $\gamma'$, corresponding to the amplitude of the incoming electric field.
These amplitudes have to be factored out if we want to really speak about reflection or transmission coefficients.

When implementing these equations, some simplifications are obvious.
For example, $\frac{E_0}{F}$ inside $K^x$ and $K^\theta$ cancels $\frac{F}{E_0}$ in $R$ and $T$;
the complex unit $j$ within $k^\theta$ combines with $j$ in $R$ and $T$ to become -1.

In these equations, $\alpha'$, $\beta'$ and $\gamma'$ are the three components of the direction of the incident electric field, satisfying $\alpha'^2 + \beta'^2 + \gamma'^2 = 1$.
The incident electric field is $(e_x, e_y, e_z) = E_0(\alpha', \beta', \gamma')$.
In order to rewrite these equations in a matrix form that depends on $e_x$, $e_y$ and $e_z$, one needs to split each coefficient of transmission and reflection into three, like this:
\begin{equation}
    e_{rx} = R^{xx} e_x + R^{xy} e_y + R^{xz} e_z
\end{equation}
so that we can write
\begin{equation}
    e_r = R 
    \begin{pmatrix}
        R_{xx} & R_{xy} & R_{xz} \\
        R_{yx} & R_{yy} & R_{yz} \\
        R_{zx} & R_{zy} & R_{zz}
    \end{pmatrix}
    e_i
\end{equation}
for the reflection $R$ and a similar equation for the transmission $e_t = T e_i$.
I start with $R_x$ defined in \cref{eq:houde_Rx} to get $R_{xx}$, $R_{xy}$ and $R_{xz}$.
\begin{align*}
    e_{rx} &= R^x E_0
    \\
           &= -\frac{F}{E_0}
              \frac{\lambda}{\pi d}
              \frac{1-\alpha^2}{\gamma}
              K^x
              E_0
    \\
           &= -\cancel{\frac{F}{E_0}}
              \frac{\lambda}{\pi d}
              \frac{1-\alpha^2}{\gamma}
              \cancel{\frac{E_0}{F}}
              \frac{N_x}{\Delta_x}
              \underbrace{
                  \alpha'
                  E_0
              }_{e_{ix}}
    \\
           &= -\frac{\lambda}{\pi d}
              \frac{1-\alpha^2}{\gamma}
              \frac{N_x}{\Delta_x}
              e_{ix}
\end{align*}
By identification, we find these values for $R_{xx}$, $R_{xy}$ and $R_{xz}$.
\begin{equation}
    \left\lbrace
    \begin{aligned}
        R_{xx} &= -\frac{\lambda}{\pi d}
                  \frac{N_x}{\Delta_x}
                  \frac{1-\alpha^2}{\gamma}
        \\
        R_{xy} &= 0
        \\
        R_{xz} &= 0
    \end{aligned}
    \right.
\end{equation}
Let us continue with $R_y$.
\begin{align*}
    e_{ry}
    &= R^y E_0
    \\
    &= \frac{F}{E_0}
       \frac{\lambda}{\pi d}
       \left[
           \frac{\alpha \beta}{\gamma}
           K^x
           -
           j
           \frac{ka}{2}
           K^\theta           
       \right]
       E_0
    \\
    &= \cancel{\frac{F}{E_0}}
       \frac{\lambda}{\pi d}
       \left[
           \frac{\alpha \beta}{\gamma}
           \cancel{\frac{E_0}{F}}
           \frac{N_x}{\Delta_x}
           \alpha'
           -
           j
           \frac{ka}{2}
           (-j)
           \cancel{\frac{E_0}{F}}
           \frac{N_\theta}{\Delta_\theta}
           (\gamma' \beta - \beta' \gamma)           
       \right]
       E_0
    \\
    &= \frac{\lambda}{\pi d}
       \left[
           \frac{\alpha \beta}{\gamma}
           \frac{N_x}{\Delta_x}
           \alpha'
           +
           \frac{ka}{2}
           \frac{N_\theta}{\Delta_\theta}
           \gamma
           \beta'
           -
           \frac{ka}{2}
           \frac{N_\theta}{\Delta_\theta}
           \beta
           \gamma'
       \right]
       E_0
    \\
    &= \frac{\lambda}{\pi d}
       \frac{\alpha \beta}{\gamma}
       \frac{N_x}{\Delta_x}
       \underbrace{E_0 \alpha'}_{e_{ix}}
       +
       \frac{\lambda}{\pi d}
       \frac{ka}{2}
       \frac{N_\theta}{\Delta_\theta}
       \gamma
       \underbrace{E_0 \beta'}_{e_{iy}}
       -
       \frac{\lambda}{\pi d}
       \frac{ka}{2}
       \frac{N_\theta}{\Delta_\theta}
       \beta
       \underbrace{E_0 \gamma'}_{e_{iz}}
\end{align*}
\begin{equation}
    \left\lbrace
    \begin{aligned}
        R_{yx}
        &=
        \phantom{-}
        \frac{\lambda}{\pi d}
        \frac{N_x}{\Delta_x}
        \frac{\alpha \beta}{\gamma}
        \\
        R_{yy}
        &=
        \phantom{-}
        \frac{\lambda}{\pi d}
        \frac{N_\theta}{\Delta_\theta}
        \frac{ka}{2}
        \gamma
        \\
        R_{yz}
        &=
        -
        \frac{\lambda}{\pi d}
        \frac{N_\theta}{\Delta_\theta}
        \frac{ka}{2}
        \beta
    \end{aligned}
    \right.
\end{equation}
Same thing for $R_z$.
\begin{align*}
    e_{rz} &= R^z E_0
    \\
    &=
    -
    \frac{F}{E_0}
    \frac{\lambda}{\pi d}
    \left[
        \alpha K^x
        +
        j
        \frac{\beta}{\gamma}
        \frac{ka}{2}
        k^\theta
    \right]
    E_0
    \\
    &=
    -
    \cancel{\frac{F}{E_0}}
    \frac{\lambda}{\pi d}
    \left[
        \alpha
        \cancel{\frac{E_0}{F}}
        \frac{N_x}{\Delta_x}
        \alpha'
        +
        j
        \frac{\beta}{\gamma}
        \frac{ka}{2}
        (-j)
        \cancel{\frac{E_0}{F}}
        (\gamma' \beta - \beta' \gamma)
        \frac{N_\theta}{\Delta_\theta}
    \right]
    E_0
    \\
    &=
    -
    \frac{\lambda}{\pi d}
    \alpha
    \frac{N_x}{\Delta_x}
    \underbrace{E_0 \alpha'}_{e_{ix}}
    +
    \frac{\lambda}{\pi d}
    \frac{\beta}{\gamma}
    \frac{ka}{2}
    \frac{N_\theta}{\Delta_\theta}
    \gamma
    \underbrace{E_0 \beta'}_{e_{iy}}
    -
    \frac{\lambda}{\pi d}
    \frac{\beta}{\gamma}
    \frac{ka}{2}
    \frac{N_\theta}{\Delta_\theta}
    \beta
    \underbrace{E_0 \gamma'}_{e_{iz}}
\end{align*}
\begin{equation}
    \left\lbrace
    \begin{aligned}
        R_{zx}
        &=
        -
        \frac{\lambda}{\pi d}
        \frac{N_x}{\Delta_x}
        \alpha
        \\
        R_{zy}
        &=
        \phantom{-}
        \frac{\lambda}{\pi d}
        \frac{N_\theta}{\Delta_\theta}
        \frac{ka}{2}
        \beta
        \\
        R_{zz}
        &=
        -
        \frac{\lambda}{\pi d}
        \frac{N_\theta}{\Delta_\theta}
        \frac{ka}{2}
        \frac{\beta^2}{\gamma}
    \end{aligned}
    \right.
\end{equation}
We have the reflection matrix.
Now, we compute the transmission matrix.
\begin{align*}
    e_{tx} &= T^x E_0
    \\
    &= (\alpha' + R^x) E_0
    \\
    &= \underbrace{\alpha' E_0}_{e_{ix}}
       -
       \frac{\lambda}{\pi d}
       \frac{1 - \alpha^2}{\gamma}
       \frac{N_x}{\Delta_x}
       \underbrace{\alpha' E_0}_{e_{ix}}
    \\
    &= \left(
           1
           -
           \frac{\lambda}{\pi d}
           \frac{1 - \alpha^2}{\gamma}
           \frac{N_x}{\Delta_x}
       \right)
       e_{ix}
\end{align*}
\begin{equation}
    \left\lbrace
    \begin{aligned}
        T_{xx}
        &= 1
           -
           \frac{\lambda}{\pi d}
           \frac{N_x}{\Delta_x}
           \frac{1 - \alpha^2}{\gamma}
        \\
        T_{xy} &= 0
        \\
        T_{xz} &= 0
    \end{aligned}
    \right.
\end{equation}

\begin{align*}
    e_{ty} &= T^y E_0
    \\
    &=
    \left(
        \beta'
        +
        \frac{F}{E_0}
        \frac{\lambda}{\pi d}
        \left[
            \frac{\alpha \beta}{\gamma}
            K^x
            +
            j
            \frac{ka}{2}
            K^\theta
        \right]
    \right)
    E_0
    \\
    &=
    \left(
        \beta'
        +
        \cancel{\frac{F}{E_0}}
        \frac{\lambda}{\pi d}
        \left[
            \frac{\alpha \beta}{\gamma}
            \cancel{\frac{E_0}{F}}
            \frac{N_x}{\Delta_x}
            \alpha'
            +
            j
            \frac{ka}{2}
            (-j)
            \cancel{\frac{E_0}{F}}
            \frac{N_\theta}{\Delta_\theta}
            (\gamma' \beta - \beta' \gamma)
        \right]
    \right)
    E_0
    \\
    &=
    \left(
        \beta'
        +
        \frac{\lambda}{\pi d}
        \frac{\alpha \beta}{\gamma}
        \frac{N_x}{\Delta_x}
        \alpha'
        -
        \frac{\lambda}{\pi d}
        \frac{ka}{2}
        \frac{N_\theta}{\Delta_\theta}
        \gamma
        \beta'
        +
        \frac{\lambda}{\pi d}
        \frac{ka}{2}
        \frac{N_\theta}{\Delta_\theta}
        \beta
        \gamma'
    \right)
    E_0
    \\
    &=
    \frac{\lambda}{\pi d}
    \frac{\alpha \beta}{\gamma}
    \frac{N_x}{\Delta_x}
    \underbrace{E_0 \alpha'}_{e_{ix}}
    +
    \left(
        1
        -
        \frac{\lambda}{\pi d}
        \frac{ka}{2}
        \frac{N_\theta}{\Delta_\theta}
        \gamma
    \right)
    \underbrace{E_0 \beta'}_{e_{iy}}
    +
    \frac{\lambda}{\pi d}
    \frac{ka}{2}
    \frac{N_\theta}{\Delta_\theta}
    \beta
    \underbrace{E_0 \gamma'}_{e_{iz}}
\end{align*}
\begin{equation}
    \left\lbrace
    \begin{aligned}
        T_{yx}
        &= \frac{\lambda}{\pi d}
           \frac{N_x}{\Delta_x}
           \frac{\alpha \beta}{\gamma}
        \\
        T_{yy}
        &= 1
           -
           \frac{\lambda}{\pi d}
           \frac{N_\theta}{\Delta_\theta}
           \frac{ka}{2}
           \gamma
        \\
        T_{yz}
        &= \frac{\lambda}{\pi d}
           \frac{N_\theta}{\Delta_\theta}
           \frac{ka}{2}
           \beta
    \end{aligned}
    \right.
\end{equation}

\begin{align*}
    e_{tz} &= T^z E_0
    \\
    &=
    \left(
        \gamma' +
        \frac{F}{E_0}
        \frac{\lambda}{\pi d}
        \left[
            \alpha K^x - j \frac{\beta}{\gamma} \frac{ka}{2} K^\theta
        \right]
    \right)
    E_0
    \\
    &=
    \left(
        \gamma' +
        \cancel{\frac{F}{E_0}}
        \frac{\lambda}{\pi d}
        \left[
            \alpha
            \cancel{\frac{E_0}{F_0}}
            \frac{N_x}{\Delta_x}
            \alpha'
            -
            j
            \frac{\beta}{\gamma}
            \frac{ka}{2}
            (-j)
            \cancel{\frac{E_0}{F}}
            \frac{N_\theta}{\Delta_\theta}
            (\gamma' \beta - \beta' \gamma)
        \right]
    \right)
    E_0
    \\
    &=
    \left(
        \gamma' +
        \frac{\lambda}{\pi d}
        \left[
            \alpha
            \frac{N_x}{\Delta_x}
            \alpha'
            +
            \frac{\beta}{\gamma}
            \frac{ka}{2}
            \frac{N_\theta}{\Delta_\theta}
            \gamma
            \beta'
            -
            \frac{\beta}{\gamma}
            \frac{ka}{2}
            \frac{N_\theta}{\Delta_\theta}
            \beta
            \gamma'
        \right]
    \right)
    E_0
    \\
    &=
    \frac{\lambda}{\pi d}
    \alpha
    \frac{N_x}{\Delta_x}
    \underbrace{E_0 \alpha'}_{e_{ix}}
    +
    \frac{\lambda}{\pi d}
    \frac{\beta}{\gamma}
    \frac{ka}{2}
    \frac{N_\theta}{\Delta_\theta}
    \gamma
    \underbrace{E_0 \beta'}_{e_{iy}}
    +
    \left(
        1
        -
        \frac{\lambda}{\pi d}
        \frac{\beta}{\gamma}
        \frac{ka}{2}
        \frac{N_\theta}{\Delta_\theta}
        \beta
    \right)
    \underbrace{E_0 \gamma'}_{e_{iz}}
\end{align*}
\begin{equation}
    \left\lbrace
    \begin{aligned}
        T_{zx}
        &= \frac{\lambda}{\pi d}
           \frac{N_x}{\Delta_x}
           \alpha
        \\
        T_{zy}
        &= \frac{\lambda}{\pi d}
           \frac{N_\theta}{\Delta_\theta}
           \frac{ka}{2}
           \frac{\beta}{\gamma}
           \gamma
        \\
        T_{zz}
        &= 1
           -
           \frac{\lambda}{\pi d}
           \frac{N_\theta}{\Delta_\theta}
           \frac{ka}{2}
           \frac{\beta}{\gamma}
           \beta
    \end{aligned}
    \right.
\end{equation}

Another simplification:
\begin{equation}
    k = 2\pi / \lambda
    \quad \Rightarrow \quad
    \frac{\lambda}{\pi d} \frac{ka}{2}
    =
    \frac{\lambda}{\pi d} \frac{2\pi a}{2\lambda}
    =
    \frac{a}{d}
\end{equation}

\begin{equation}
    R =
    \begin{pmatrix}
        -\frac{\lambda}{\pi d}
        \frac{N_x}{\Delta_x}
        \frac{1 - \alpha^2}{\gamma}
        &
        0
        &
        0
        \\
        \frac{\lambda}{\pi d}
        \frac{N_x}{\Delta_x}
        \frac{\alpha \beta}{\gamma}
        &
        \frac{\lambda}{\pi d}
        \frac{N_\theta}{\Delta_\theta}
        \frac{ka}{2}
        \gamma
        &
        -
        \frac{a}{d}
        \frac{N_\theta}{\Delta_\theta}
        \beta
        \\
        -
        \frac{\lambda}{\pi d}
        \frac{N_x}{\Delta_x}
        \alpha
        &
        \frac{a}{d}
        \frac{N_\theta}{\Delta_\theta}
        \beta
        &
        -
        \frac{a}{d}
        \frac{N_\theta}{\Delta_\theta}
        \frac{\beta^2}{\gamma}
    \end{pmatrix}
\end{equation}
\begin{equation}
    T =
    \begin{pmatrix}
        1 -
        \frac{\lambda}{\pi d}
        \frac{N_x}{\Delta_x}
        \frac{1 - \alpha^2}{\gamma}
        &
        0
        &
        0
        \\
        \frac{\lambda}{\pi d}
        \frac{N_x}{\Delta_x}
        \frac{\alpha \beta}{\gamma}
        &
        1 -
        \frac{a}{d}
        \frac{N_\theta}{\Delta_\theta}
        \gamma
        &
        \frac{a}{d}
        \frac{N_\theta}{\Delta_\theta}
        \beta
        \\
        \frac{\lambda}{\pi d}
        \frac{N_x}{\Delta_x}
        \alpha
        &
        \frac{a}{d}
        \frac{N_\theta}{\Delta_\theta}
        \beta
        &
        1 -
        \frac{a}{d}
        \frac{N_\theta}{\Delta_\theta}
        \frac{\beta^2}{\gamma}
    \end{pmatrix}
\end{equation}
This reflection and transmission matrix are Jones matrices.
They must find their place inside the scattering matrix.
But before, we must apply rotation matrices to them in order to account for the arbitrary orientation of the grid.
Indeed, these Jones matrices are valid for grid lying in the $(x, y)$ plane, with the wires along $x$.
Doing this follows the usual method for rotating Jones matrices described by \cref{eq:jones_rotation_using_transpose}.





%#############################################################################

\section{Simple systems}
\label{sec:simple_systems}

The idea is to show that it works and makes sense.



%=============================================================================

\subsection{The simplest cavity}
\label{sec:the_simplest_cavity}

\Cref{fig:simple_cavity_principle} illustrates how three networks can represent a simple cavity: two interfaces separated by some space.
The three networks all have two ports numbered according to \cref{fig:simple_cavity_principle}.
Ports 2 and 3 are coupled, and so are ports 4 and 5.
Ports 1 and 6 are open to the outside world.

\begin{figure}[hbtp]
    \centering
    \input{\MyGraphicsPath simple_cavity_principle.pdf_tex}
    \caption{Simple cavity, principle.}
    \caption*{
        Two reflective surfaces facing each other form a cavity.
        Here, the surfaces are defined by the interfaces (vertical dotted lines)
        between regions of space of different refractive indices $n_1$ and $n_2$.
        According to the Fresnel equations~\eqref{eq:fresnel_normal}, these interfaces
        reflect and transmit part of the incident radiation (black arrows).
        As a result, a standing wave is formed inside the cavity.
        That standing wave is the superposition of an infinity of traveling waves
        interfering with each other.
        We can model such a system with three two-ports networks:
        one for each interface and one for the space between them.
        The numbers 1 to~6 are our arbitrary labels for the ports.
    }
    \label{fig:simple_cavity_principle}
\end{figure}
\begin{figure}[hbtp]
    \centering
    \includegraphics{simple_cavity_direct}
    \caption{Simple cavity, model result.}
    \caption*{
        The blue curve, on top, corresponds to the power coupling of port~6,
        that is the transmission through the cavity.
        The red curve, on the bottom, corresponds to the power coupling of port~1,
        that is the reflection on the cavity.
    }
    \label{fig:simple_cavity_direct}
\end{figure}
\begin{figure}[hbtp]
    \centering
    \includegraphics{simple_cavity_fft}
    \caption{Simple cavity, Fourier transform of the model result.}
    \caption*{
        The transmitted (blue) and reflected (red) wave show the same amount of modulation
        introduced by the cavity.
        The fundamental is at a period $F=c/2d$ with $c$ the speed of light in the cavity
        and $d$ the length of the cavity.
        An harmonic shows at $F=c/4d$,
        proving that the modulation is not perfectly sinusoidal.
        This spectrum was obtained by running the model for 4001 frequencies
        over a~\SI{64}{\giga\hertz} range,
        multiplying the result by a hanning window and taking a fast Fourier transform.
    }
    \label{fig:simple_cavity_fft}
\end{figure}

\Cref{fig:simple_cavity_direct} presents the result of the model for the following parameters:
\begin{itemize}
    \item refractive index $n_1=1.5$,
    \item refractive index $n_2=1.0$,
    \item distance between the interfaces $d=\SI{0.5}{\meter}$,
    \item frequency $f$ from \SIrange{500}{504}{\giga\hertz}.
\end{itemize}
The scattering matrices of the interfaces are derived from~\vref{eq:s_interface_normal} using $n_1$ and $n_2$.
The scattering matrix of the space between the interfaces is derived from~\vref{eq:scattering_distance} using $d$, $n_1$ and $f$.



%-----------------------------------------------------------------------------

\subsubsection{Periodicity}
According to \cref{eq:cavity_period}, we expect a periodicity of $F=c/2d$ where $c=n c_0$ is the speed of light in the cavity and $d$ the length of the cavity.
With $c_0\approx \SI{2.998e8}{\meter\per\second}$, $n=1$ and $d=\SI{0.5}{\meter}$, we get $F \approx \SI{299.8}{\mega\hertz}$.

Our model agrees with our expectation, as shown by the fourrier transform in~\cref{fig:simple_cavity_fft}.



%-----------------------------------------------------------------------------

\subsubsection{Energy conservation}
Examination of the results displayed in~\cref{fig:simple_cavity_direct} reveals that energy is neither created nor destroyed.
The power is always positive, and the sum of the transmitted (blue) and reflected (red) power always equal 1 within numerical noise.
The power that is not transmitted through the cavity is reflected back to the source.



%=============================================================================

\subsection{Thin film beam splitter}

A thin dielectric film can be used as a beam splitter, as illustrated in~\cref{fig:beam_splitter_principle}.
\begin{figure}[hbtp]
    \centering
    \input{\MyGraphicsPath beam_splitter_principle.pdf_tex}
    \caption{Beam splitter, principle.}
    \label{fig:beam_splitter_principle}
\end{figure}

We can model a thin film using the principle described in \cref{sec:the_simplest_cavity}.
In the case of a film, the two interfaces are separated by a few micrometers only.
We also need to take into account the \SI{45}{\degree} angle of incidence.

%-----------------------------------------------------------------------------

\subsubsection{Modeling the thin film}

Let us model a thin film of biaxially-oriented polyethylene terephthalate or ``boPET'', more comonly known under trade-mame ``Mylar''.
The table entry for ``PETP'' in the article of \citeauthor{lamb1996miscellaneous}~\cite{lamb1996miscellaneous} suggests that we choose 1.83 for the refractive index of the film at~\SI{500}{\giga\hertz}.
This 1.83 corresponds to the real part~$n$ of the refractive index~$\hat{n}=n-ik$.
The imaginary part~$k$ is given by the ``$\tan \delta$'' column of the table and
the equation (6) of that article, which links $\tan \delta$ to $k$: $\tan \delta = 2k/n$.
Therefore, $k = n \tan \delta / 2$.
With $n=1.83$ and $\tan \delta = 0.020$, we have $k=0.018$.
The complex refractive index of our thin film is $1.83-0.018i$.





Then I introduce a thin film beam splitter (that's already a heterodyne telescope when we think of it).  I show that it adds a slope to the previous pattern.  I show that it's actually a very slow standing wave, the period of which is consistent with infinite reflections inside the thin film.  The FFT should show that there are two picks close to the cavity-period: they are due to the fact that there are now two cavities: one using reflection on the near side, one on the far side of the beam splitter.

All of this makes sense, we are confident that the model does its job properly.
