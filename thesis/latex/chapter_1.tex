\section{Coherence and interference}
Explain coherence.

Introduce coherence time, coherence length.

Coherence makes you care about the phase, and also about the polarization.

Explain interference.

Show that incoherent systems do not have to care.
Incoherent system: no phase information, just power.
Incoherent system can still have polarization, but they can be unpolarized.
Coherent system are always polarized.



\section{Cavities}
Explain cavity.
Explain and show that the modulation is periodic.
Low Q and high Q: sine is an approximation that works only at low Q.


\section{The Heterodyne Instrument for the Far Infrared}

Keywords: Heterodyne, double sideband, sideband ratio.

The HIFI instrument on the Herschel Space Observatory~\cite{AA_518_L1} is a heterodyne receiver that operates at frequencies between \SI{480}{\giga\hertz} and \SI{1910}{\giga\hertz},
producing spectra with a resolution ranging from \SI{1}{\mega\hertz} to \SI{125}{\kilo\hertz}~\cite{AA_518_L6}.
This high frequency resolution enables the astronomers to study the chemistry of a wide range of phenomena, from planetary atmospheres to star forming regions.

At such a high resolution, the thermal noise of the astronomical source, the calibration black bodies and the local oscillator (LO) qualify as ``narrow band Gaussian noise signals''~\cite{siegman1986lasers}.
They have a coherence time~$\tau$ equal to the inverse of their bandwidth~$\tau=1/\Delta f$.
A bandwidth of~\SI{1}{\mega\hertz} results in a coherence time of~\SI{1}{\micro\second} equivalent to a coherence length of~\SI{300}{\meter}.
This is a hundred times the longest distance inside HIFI.
Therefore, in HIFI, the signals from the LO, the calibration sources and the sky are coherent.

Show spectra with standing wave patterns.



\section{The sideband ratio problem}

Explain how we perform the flux calibration.
Explain how it cannot cancel all the standing waves.
Explain how the dual sideband aspect makes things difficult: many more variables but not more equations.
This is where the sideband ratio comes into play: it connects the LSB and USB gains, reducing the number of independant variables in the system.

Because of the lack of equations, the sideband ratio cannot be retreived from the data and, instead, must be assumed.
This is the motivation of my model.

Show that, by looking at their period, we determined a while ago the cavities responsible.
Point to the cavities on drawing.
Tells that it imposes welcome constraints on our model.



\section{Modeling standing waves}
It is not really about modelling standing waves, it is about modeling coupling.
Standing waves are taken care of during the process.

In brief, the general ideas of how my model works, and why.
This includes some confession regarding simplifications.

Conclude that the method in this thesis can be applied to any single-mode coherent system.
Lasers are good candidates.



\section{Applications to astronomy}
