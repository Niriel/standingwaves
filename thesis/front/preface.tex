\cleardoublepage
%\begin{refsection}
\chapter{Preface}
\label{sec:preface}

Advances in imaging and spectroscopy technology are always eagerly welcomed.
Two groups in particular are quick to adopt and improve upon new techniques: the military, who look down and keep secrets, and astronomers, who look up and share knowledge.

For example, Willard S.\ Boyle and George E.\ Smith invented the CCD in~1969%
\footnote{
   They were awarded the Nobel Prize in Physics 2009 for its invention~\parencite{nobel2009}.
}%
.
This ``charge-coupled devive'' was a semiconductor circuit that converts photons to electric charges and was intended to be used as a memory cell.
In 1976, only seven years later,
Caltech submitted the proposal of the Wide Field and Planetary Camera,
based on CCD technology,
to be part of the Hubble Space Telescope (HST)~\parencite{leckrone1980space}.
In December of the same year, a $800\times800$~pixels CCD camera was orbiting Earth: the American National Reconnaissance Office launched the KH-11 Kennan, a spy satellite similar in design to the HST~\parencite[see references in][]{wiki:kh11kennan}.

Astronomers, for whom every single photon matters, have worked with manufacturers to reduce the noise of CCD down to its minimal quantum-limited level: CCD technology is now virtually perfect~\parencite{mackay2010}.
CCD and related technologies such as CMOS-based Active-Pixel Sensors are nowadays ubiquitous, embedded inside mobile phones, computers, security cameras, medical devices and bar-code readers.
Nevertheless, CCD technology suffers from one considerable limitation: it is sensitive only to photons whose wavelength is between~\SI{0.2}{\micro\meter} (mid-ultraviolet) and~\SI{1.0}{\micro\meter} (near-infrared) and operates best at visible wavelengths.

There is much more to electromagnetic radiation than visible light.
Since electromagnetic radiation is almost the only way to get data from beyond the solar system%
\footnote{
    Astronomers also measure cosmic rays (relativistic protons and electrons) and neutrinos from supernovae, the sun or other objects.
    Gravitational waves may soon become a new source of information;
    there is indirect evidence of their existence (from the spin-down of double pulsars) and missions like LISA Pathfinder may detect them directly before~2017.
}%
, astronomers are resolved to explore its entire spectrum.

A significant fraction of the universe consists of gas and dust that is too cold to radiate visible light.
This cold material is associated with the earliest stages of galaxy evolution, star formation, protoplanetary disks and the atmosphere of comets.
%
The far-infrared region of the electromagnetic spectrum covers wavelengths
between \SI{30}{\micro\meter} and \SI{300}{\micro\meter},
which correspond to frequencies
between \SI{1}{\tera\hertz} and \SI{10}{\tera\hertz}.
Atoms and ions have most of their fine-structure transitions within that range.
%
The submillimeter region extends from~\SI{300}{\micro\meter} to~\SI{3}{\milli\meter} (\num{0.1} to \SI{1}{\tera\hertz}) and contains many molecular transition lines.
%
In addition, Planck's law peaks in the far-infrared and submillimeter for temperatures below a few hundred kelvin.
%
Consequently, these are regions of choice to study the chemistry and dynamics of the cold universe.

%\begin{figure}
%    \centering
%    \begin{tikzpicture}[scale = 0.7]
%        \draw[->] (0,0) -- (10.5,0) node[anchor=west] {frequency [\si{\tera\hertz}]};
%        \foreach \x in {0,1,...,10} {
%            \draw (\x,0) -- (\x,-0.3) node[anchor=north] {$10^{\x}$};
%        }
%    \end{tikzpicture}
%    \caption{The terahertz gap.}
%    \label{fig:terahertzgap}
%\end{figure}

The far-infrared and submillimeter domain resides between
the optical world of mirrors and lenses on one side,
and the microwave world of waveguides and antennas on the other side.
Optical systems can often be modeled with refractive indices and infinitely-thin rays carrying power%
\footnote{
    Except when phase matters, as it does for interferometry,
    lasers, and consequently holography and optical-fiber communication.
}
while microwave devices are best understood with concepts like hybrid modes, impedance matching and voltage.
Optics and microwave do not mix,
a problem which is more fundamental than a paradigm clash as it is rooted in the very structure of physical materials.
In 2015 still, useful power generation and receiver technologies are inefficient and impractical at terahertz frequencies;
mass production of devices in this range and operation at room temperature remains mostly unfeasible.

Despite the considerable difficulties, astronomers were determined to fill this so-called ``Terahertz Gap''.
Terahertz astronomy became a priority in the early 1980's.
In 1985, ESA selected FIRST (who would be later known as the Herschel Space Observatory) as a ``cornerstone'' mission of its ``horizon 2000'' program.
In 1988, NASA established a ``Center for Space Terahertz Technology'' at the University of Michigan.
In 1989, the Cosmic Background Explorer (COBE) was launched.
In 1990 was organized the first ``International Symposium on Space Terahertz Technology'' (ISSTT).
This ISSTT has been held every year since and has maintained the synergy between astronomers, space engineers and all terahertz scientists worldwide.
The ISSTT proceedings are published on the NRAO website~\parencite{nrao:isstt}.
In 1990, stratospheric or space missions like IRTS, ISO, PRONAOS, SIRTF (Spitzer), SMMM, SOFIA and SWAS were either ready to launch or on in their design phase.
The research and development for the most ambitious projects took decades.

I was granted the honor to present my work during the 24\textsuperscript{th}
edition of the ISSTT in 2013%
\footnote{
    Unfortunately, ISSTT did not publish proceedings that year.
}%
, which was held in Groningen (The Netherlands) and organized by the University of Groningen (RUG) and the Netherlands Institute for Space Research (SRON).
The date and the place were wisely chosen:
2013 coincided with the end of the four-year flight of the Herschel Space Observatory, nearly \SI{30}{years} after its inception;
and SRON-Groningen was the Principal Investigator institute for Herschel's high-resolution spectrometer, HIFI~\parencite{AA_518_L6}.
SRON coordinated the efforts of~25 institutes over a dozen countries, which created the technology that gave the astronomical community an instrument that could cover four frequency octaves (from \num{480} to \SI{1910}{\giga\hertz}) with a spectral resolution between \num{0.5e6} and \num{15e6} and a near-quantum-limited sensitivity.

The problem of standing waves and optical interferences in HIFI, which motivated my work, is not the result of an oversight or a poor design; precautions were taken from the very beginning of the project to minimize them.
On the contrary, worrying about these remaining interferences is a luxury only made possible by the low noise and high stability of the system~\parencite{AA_537_A17}.

Research and development are going on.
One can find examples both on the front of spectroscopy
--- sideband-separating mixers performing at \SI{700}{\giga\hertz} are now available~\parencite{khudchenko2011first} --- and imaging --- with the tenfold increase in the sensitivity of Transition Edge Sensors~\parencite{jackson2012spica}.


The terahertz gap may still not be filled today, but astronomers have bridged it.






%#############################################################################
%\clearpage
%\printbibliography[heading=subbibliography]
%\end{refsection}
