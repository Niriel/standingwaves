\usepackage[T1]{fontenc}
\usepackage[utf8]{inputenc}
\usepackage{lmodern}

\usepackage{amsmath}
    % Brings the align environment for lining up equations on the =.

% According to
% http://tex.stackexchange.com/questions/96350/problem-with-algorithmic-and-hyperref
% if I want to have hyperref working with algorithm, I must
% first load float, then hyperref, then algorithm.
% Putting float here solves the TOC links to the algorithms, otherwise they'd all point
% to page 1.
% Also, putting float-algo-hyper creates a massive amount of warnings for pretty much
% every algo command like \State.
\usepackage{float}


\usepackage{makeidx}
    % Needed to build the index.

\usepackage[toc,page]{appendix}
    % Brings \begin{appendices}.
    % toc option creates an "appendices" entry in the table of content.
    % page option creates a page before the first appendix chapter.

\usepackage{amssymb}
    % For the number set symbols via \mathbb.

\usepackage[version=3]{mhchem}
    % For chemical formulas.
    % Brings \ce.

%\usepackage[mediumspace,mediumqspace,squaren]{SIunits}
\usepackage{siunitx}
    % Otherwise I can't write the mu symbol for micrometers.
    % It also brings me the \degree symbol, woo!
    % No decibel though, I need to make this one myself.

\usepackage[makeroom]{cancel}
    % For striking out bits of equations that simplify.

\usepackage{graphicx}
    % Can't have pictures/photos/figures from files without that.
    % Note that vanilla latex will only accept eps.


\usepackage{import}
    % Facilitates the organization of the book.  I can place each chapter in its own
    % directory, along with its images.


\usepackage[font=footnotesize, labelfont=bf]{caption}
    % So that I can add long captions under figures.
    % It provides me with \caption* that does not appear in the list of
    % figures but formats the text like \caption does.  It also lets me
    % use line breaks in the caption, and even bullet/enumerated lists.

\usepackage{subcaption}
    % Lets me have labels such as a) b) c) inside a figure environment.

\usepackage[table]{xcolor}
    % xcolor gives color.
    % The 'table' parameter loads another package that lets me set
    % alternating colors for the rows of a table.

\usepackage{booktabs}
    % For professional-looking tables.
    % Brings the \toprule, \midrule and \bottomrule.
    % Remember not to use vertical rules in tables: they look cheap.

\usepackage{multirow}

\usepackage{tabularx}

\usepackage{stmaryrd}
    % For the llbracket and rrbracket to denote integer intervals.
    
\usepackage{todonotes}
    % Big post-its, handy while writing.
    
\usepackage[]{biblatex}
    % Fantastic bibliography manager that I'll use just for its
    % \citetitle command.

% As a rule of thumb, hyperref goes last.  Only a few packages go after.
% cleveref is a well known one.  bookmark is another one.

% According to
% http://tex.stackexchange.com/questions/53191/impact-of-hyperref-when-varioref-and-cleveref-are-used/53193#53193
% one must first load varioref, then hyperref, then cleveref.
% Failure to do so results in lots of hyperrefs warnings and links that do not point to the right place.
\usepackage{varioref}
    % For fancier references that also tell the page number.

\usepackage[plainpages=false, hypertexnames=false]{hyperref}
    % Requires direct pdf output, therefore all figures should be pdf too.
    % This creates hyperlinks all over the place to jump to figures, references,
    % chapters and all.
    % hypertexnames=false makes the package dumber, allowing it to work even if
    % there are several chapter 1 in the document.
    %
    
\usepackage{cleveref}
    % Should make the typesetting of cross references more consistent.
    % Note that I needed to install a recent version of this package, the one I had
    % was too old to handle algorithms properly.
    % NOTE that cleveref is confused by the subcaption package.  Whenever you place
    % a label in a subfigure, cleveref hangs forever.  The label at the figure level
    % works fine though.  So, no labeling of subfigures :(.

% According to
% http://tex.stackexchange.com/questions/113719/cleveref-fails-to-reference-algorithms
% Algorithm must be loaded before hyperref.  Well, it is not true.
\usepackage[chapter]{algorithm}
    % To float algorithms.
\usepackage{algpseudocode}
    % For pseudocode.

\crefname{appendix}{appendix}{appendices}
\Crefname{appendix}{Appendix}{Appendices}

\newcommand{\decibel}{dB}
\newcommand{\equaldef}{\stackrel{\text{\tiny def}}{=}}
\newcommand{\transp}{^\top}
\newcommand{\norm}[1]{\left\| #1 \right\|}
\newcommand{\abs}[1]{\left| #1 \right|}
\newcommand{\vect}[1]{\mathbf{#1}}
\newcommand{\atantwo}{\textrm{atan2}}
\newcommand{\vangle}[2]{\left(#1, #2\right)}  % Angle between vectors.
\newcommand{\transition}[3]{$\ce{#1}_{#2 \rightarrow #3}$}
