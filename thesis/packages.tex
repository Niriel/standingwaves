\usepackage[T1]{fontenc}
\usepackage[utf8]{inputenc}
\usepackage{lmodern}

%==================================  FLOATS  =================================

% According to
% http://tex.stackexchange.com/questions/96350/problem-with-algorithmic-and-hyperref
% if I want to have hyperref working with algorithm, I must
% first load float, then hyperref, then algorithm.
% Putting float here solves the TOC links to the algorithms, otherwise they'd
% all point to page 1.
% Also, putting float-algo-hyper creates a massive amount of warnings for pretty much
% every algo command like \State.
\usepackage{float}

\usepackage{placeins}
    % Brings \FloatBarrier command. It forces Tex to typeset all remaining floats
    % at that point and doesn't include a \clearpage afterwards.
    % Nice to flush all the floats before beginning a new section.

\usepackage[font=footnotesize,labelfont=bf]{caption}
    % So that I can add long captions under figures.
    % It provides me with \caption* that does not appear in the list of
    % figures but formats the text like \caption does.  It also lets me
    % use line breaks in the caption, and even bullet/enumerated lists.

\usepackage{subcaption}
    % Lets me have labels such as a) b) c) inside a figure environment.
    % Unfortunately something is broken, I cannot give labels to these
    % subcaptions when hyperef is active, the pdf compiler hangs forever.
    % Seems to be a common problem.

\usepackage{rotating}
    % Used to display long tables in landscape mode.

%http://tex.stackexchange.com/questions/38972/how-to-forbid-latex-to-put-text-between-figures
\renewcommand\topfraction{0.85}
\renewcommand\bottomfraction{0.85}
\renewcommand\textfraction{0.1}
\renewcommand\floatpagefraction{0.85}

%===============================  BACKMATTER  ================================

\usepackage{makeidx}
    % Needed to build the index.

\usepackage[toc,page]{appendix}
    % Brings \begin{appendices}.
    % toc option creates an "appendices" entry in the table of content.
    % page option creates a page before the first appendix chapter.

\usepackage[sorting=none,citestyle=authoryear]{biblatex}
    % Fantastic bibliography manager that I'll use just for its
    % \citetitle command.
    % For sorting orders, see
    % https://www.sharelatex.com/blog/2013/07/31/getting-started-with-biblatex.html

%==============================  SUBDIRECTORIES  =============================

\usepackage{import}
    % Facilitates the organization of the book.  I can place each chapter in its own
    % directory, along with its images.
    % Brings /subimport{directory}{tex_file_name_without_extension}.

%===================================  COLOR  =================================

\usepackage[table]{xcolor}
    % The 'table' parameter loads another package that lets me set
    % alternating colors for the rows of a table.
    % xcolor also imports color, which is useful for importing figures saved as
    % PDF+TEX files generated by inkscape.
    
    % Must be loaded before todonotes: todonotes loads color without the [table]
    % option.

%=================================  DRAFTING  ================================

\ifDraft
    \usepackage{todonotes}
        % Big post-its, handy while writing.
\else
    \usepackage[disable]{todonotes}
\fi

%===============================  MATHEMATICS  ===============================

\usepackage{amsmath}
    % Brings the align environment for lining up equations on the =.

\usepackage{bm}
    % For bold mathematics.
    % An advantage of using \bm over \mathbf is that \bm keeps the
    % letters slanted in maths.  I use bold letters for vectors,
    % I see no reason to have them typeset upright.
    % Another advantage is that it works on greek letters too.

\usepackage{amssymb}
    % For the Real/Complex/etc. number set symbols via \mathbb.

\usepackage[makeroom]{cancel}
    % For striking out bits of equations that simplify.

\usepackage{stmaryrd}
    % For the llbracket and rrbracket to denote integer intervals.

% Fix the spacing introduced by \left and \right in formulas.
\let\originalleft\left
\let\originalright\right
\renewcommand{\left}{\mathopen{}\mathclose\bgroup\originalleft}
\renewcommand{\right}{\aftergroup\egroup\originalright}

\newcommand{\equaldef}{\stackrel{\text{\tiny def}}{=}}
\newcommand{\transp}{^\top}                     % Tranposed of a matrix.
\newcommand{\norm}[1]{\left\| #1 \right\|}      % Norm of a vector.
\newcommand{\abs}[1]{\left| #1 \right|}         % Absolute value of a scalar.
\newcommand{\cp}[1]{\hat{#1}}                   % Complex object.
\newcommand{\timeavg}[1]{\langle #1 \rangle}
\newcommand{\Timeavg}[1]{\left\langle #1 \right\rangle}
\newcommand{\vect}[1]{\bm{#1}}                  % Real vector.
\newcommand{\vectcp}[1]{\cp{\vect{#1}}}         % Complex vector.
\newcommand{\vectu}[1]{\dot{\vect{#1}}}         % Unit (real) vectors.
\newcommand{\matr}[1]{\mathrm{#1}}              % Real matrix.
\newcommand{\matrcp}[1]{\cp{\matr{#1}}}         % Complex matrix.
\DeclareMathOperator{\atantwo}{atan2}           % Arctan(y, x).
\newcommand{\dif}{\mathrm{d}}                   % For integrals: dx dy dz.
\newcommand{\inprod}[2]{\langle #1, #2 \rangle} % Inner product of two vectors.
\newcommand{\vangle}[2]{\left(#1, #2\right)}    % Angle between vectors.
% Used as indices:
\newcommand{\re}{r} % Real part.
\newcommand{\im}{i} % Imaginary part.

%================================  QUANTITIES  ===============================

\usepackage{siunitx}
    % Otherwise I can't write the mu symbol for micrometers.
    % It also brings me the \degree symbol, woo!
    % No decibel though, I need to make this one myself.
    
    % My version of siunitx is too old so I cannot use \DeclareSIUnit to create
    % new units.
\newcommand{\decibel}{dB}  % \DeclareSIunit doesn't work with my version.
\newcommand{\parsec}{pc}
\newcommand{\astronomicalunit}{au}  % Was ua until 2014.

%=================================  PHYSICS  =================================

% Used as indices:
\newcommand{\I}{I} % Incident.
\newcommand{\R}{R} % Reflected.
\newcommand{\T}{T} % Transmitted/refracted.

%================================  CHEMISTRY  ================================

\usepackage[version=3]{mhchem}
    % For chemical formulas.
    % Brings \ce.

\newcommand{\Jlevel}[2]{$J=#1\!\rightarrow\!#2$}
\newcommand{\transition}[3]{\ce{#1}~\Jlevel{#2}{#3}}

%==================================  LISTS  ==================================

\usepackage{enumitem}
    % Allows customization of bullet-lists and numbered-lists.
    % In particular, allows removing the insane amount of vertical space that latex
    % injects in these lists by default.
    % Usage: \begin{itemize}[noitemsep,nolistsep]
    % noitemsep removes the spacing between the items.
    % nolistsep removes the spacing between the paragraph and the list.
    % Global usage: \setlist[itemize]{noitemsep}
    % Except that this global usage doesn't work on my old installation.


%=================================  PICTURES  ================================

\usepackage{graphicx}
    % Can't have pictures/photos/figures from files without that.
    % Note that vanilla latex will only accept eps while tex2pdf accepts
    % anything except eps.  Go figure...



%=================================  TABLES  ==================================

\usepackage{booktabs}
    % For professional-looking tables.
    % Brings the \toprule, \midrule and \bottomrule.
    % Remember not to use vertical rules in tables: they look cheap.

\usepackage{multirow}
    % How isn't that even standard?

\usepackage{tabularx}
    % Brings the X column type, auto expand to fill the page.

\usepackage{array}
    % Brings >{} and <{} column modifiers to wrap the content into stuff.

%=============================  CROSS REFERENCES  ============================

% As a rule of thumb, hyperref goes last.  Only a few packages go after.
% cleveref is a well known one.  bookmark is another one.

% According to
% http://tex.stackexchange.com/questions/53191/impact-of-hyperref-when-varioref-and-cleveref-are-used/53193#53193
% one must first load varioref, then hyperref, then cleveref.
% Failure to do so results in lots of hyperrefs warnings and links that do not point to the right place.
\usepackage{varioref}
    % For fancier references that also tell the page number.

\ifPrint
    \usepackage[draft]{hyperref}
\else
    \usepackage[plainpages=false, hypertexnames=false]{hyperref}
    % Requires direct pdf output, therefore all figures should be pdf too.
    % This creates hyperlinks all over the place to jump to figures, references,
    % chapters and all.
    % hypertexnames=false makes the package dumber, allowing it to work even if
    % there are several chapter 1 in the document.
    \hypersetup{
        colorlinks,
        linkcolor={red!50!black},
        citecolor={green!40!black},
        urlcolor={blue!80!black}
    }
\fi
    
\usepackage{cleveref}
    % Should make the typesetting of cross references more consistent.
    % Note that I needed to install a recent version of this package, the one I had
    % was too old to handle algorithms properly.
    % NOTE that cleveref is confused by the subcaption package.  Whenever you place
    % a label in a subfigure, cleveref hangs forever.  The label at the figure level
    % works fine though.  So, no labeling of subfigures :(.
\crefname{appsec}{appendix}{appendices}
\Crefname{appsec}{Appendix}{Appendices}

%=============================  ALGOS AND CODE  ==============================

% According to
% http://tex.stackexchange.com/questions/113719/cleveref-fails-to-reference-algorithms
% Algorithm must be loaded before hyperref.  Well, it is not true.
\usepackage[chapter]{algorithm}
    % To float algorithms.
\usepackage{algpseudocode}
    % For pseudocode.

%===============================  SPECIAL TEXT  ==============================

\hyphenation{aniso-tro-pic iso-tro-pic impe-dance va-cuum algo-rithm algo-rithms col-linear pe-rio-dic}
\newcommand{\radex}{\textsc{radex}}
\newcommand{\Radex}{\textsc{Radex}}
